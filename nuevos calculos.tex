\documentclass{article}
\usepackage{amsmath}
\usepackage{amssymb}
\usepackage[spanish]{babel}

\begin{document}

\section*{Derivación de la restricción de longitud constante}

\subsection*{Ecuación inicial}
Dado:
\begin{align}
K^2 &= (R\cos(\theta) - D)^2 + (R\sin(\theta) - y_c)^2
\end{align}

Donde:
\begin{itemize}
    \item $K$ es constante
    \item $\frac{d\theta}{dt} = \omega_1$
    \item $y_c$ es variable
    \item $R$ es constante
    \item $D$ es constante
\end{itemize}

\subsection*{Derivación respecto al tiempo}
Derivando ambos lados respecto al tiempo:
\begin{align}
\frac{d(K^2)}{dt} &= \frac{d}{dt}\left[(R\cos(\theta) - D)^2 + (R\sin(\theta) - y_c)^2\right]
\end{align}

Como $K$ es constante:
\begin{align}
0 &= \frac{d}{dt}\left[(R\cos(\theta) - D)^2 + (R\sin(\theta) - y_c)^2\right]
\end{align}

Aplicando la regla de la cadena:
\begin{align}
0 &= 2(R\cos(\theta) - D)\frac{d}{dt}(R\cos(\theta) - D) + 2(R\sin(\theta) - y_c)\frac{d}{dt}(R\sin(\theta) - y_c)
\end{align}

Calculando las derivadas internas:
\begin{align}
\frac{d}{dt}(R\cos(\theta) - D) &= -R\sin(\theta)\frac{d\theta}{dt} = -R\sin(\theta)\omega_1 \\
\frac{d}{dt}(R\sin(\theta) - y_c) &= R\cos(\theta)\frac{d\theta}{dt} - \dot{y}_c = R\cos(\theta)\omega_1 - \dot{y}_c
\end{align}

Sustituyendo:
\begin{align}
0 &= 2(R\cos(\theta) - D)(-R\sin(\theta)\omega_1) + 2(R\sin(\theta) - y_c)(R\cos(\theta)\omega_1 - \dot{y}_c)
\end{align}

Dividiendo entre 2:
\begin{align}
0 &= -(R\cos(\theta) - D)R\sin(\theta)\omega_1 + (R\sin(\theta) - y_c)(R\cos(\theta)\omega_1 - \dot{y}_c)
\end{align}

Expandiendo:
\begin{align}
0 &= -R^2\cos(\theta)\sin(\theta)\omega_1 + DR\sin(\theta)\omega_1 \nonumber \\
  &\quad + R^2\sin(\theta)\cos(\theta)\omega_1 - R\sin(\theta)\dot{y}_c \nonumber \\
  &\quad - y_cR\cos(\theta)\omega_1 + y_c\dot{y}_c
\end{align}

Los términos $R^2\cos(\theta)\sin(\theta)\omega_1$ se cancelan, quedando:
\begin{align}
0 &= DR\sin(\theta)\omega_1 - R\sin(\theta)\dot{y}_c - y_cR\cos(\theta)\omega_1 + y_c\dot{y}_c
\end{align}

Reorganizando:
\begin{align}
\dot{y}_c(R\sin(\theta) - y_c) &= R\omega_1(D\sin(\theta) - y_c\cos(\theta))
\end{align}

\subsection*{Despeje de $\dot{y}_c$}
Dividiendo ambos lados por $(R\sin(\theta) - y_c)$:
\begin{align}
\boxed{\dot{y}_c = \frac{R\omega_1(D\sin(\theta) - y_c\cos(\theta))}{R\sin(\theta) - y_c}}
\end{align}

Esta expresión es válida siempre que $R\sin(\theta) \neq y_c$ (es decir, que el denominador no sea cero).

\subsection*{Aceleración mediante lazo vectorial}

Derivando nuevamente la ecuación de velocidad para obtener la aceleración:

\begin{equation}
\ddot{y}_c = 
R \; 
\frac{
\big[ \dot{\omega}_1 A - \omega_1^2 B - \omega_1 \dot{y}_c \cos\theta \big] D_{en}
- \omega_1 A (\dot{y}_c + R \omega_1 \cos\theta)
}{
D_{en}^2
}
\end{equation}

\noindent donde:
\begin{align*}
A &= D\sin\theta - y_c\cos\theta,\\
B &= D\cos\theta + y_c\sin\theta,\\
D_{en} &= R\sin\theta - y_c.
\end{align*}

Nota: $\dot{\omega}_1 = \ddot{\theta} = \alpha_1$ es la aceleración angular.

\subsection*{Forma expandida de la aceleración}

Expandiendo la expresión anterior:

\begin{align}
\ddot{y}_c &= R \frac{\alpha_1 A \cdot D_{en} - \omega_1^2 B \cdot D_{en} - \omega_1 \dot{y}_c \cos\theta \cdot D_{en} - \omega_1 A \dot{y}_c - R\omega_1^2 A \cos\theta}{D_{en}^2}
\end{align}

Factorizando:

\begin{align}
\ddot{y}_c &= \frac{R(R\sin\theta - y_c)\alpha_1(D\sin\theta - y_c\cos\theta)}{(R\sin\theta - y_c)^2} \nonumber \\
&\quad + \frac{R\omega_1^2[Ry_c - Dy_c\cos\theta - y_c^2\sin\theta - R(D\cos\theta + y_c\sin\theta)]}{(R\sin\theta - y_c)^2} \nonumber \\
&\quad + \frac{R\omega_1\dot{y}_c[\cos\theta(R\sin\theta - y_c) - (D\sin\theta - y_c\cos\theta)]}{(R\sin\theta - y_c)^2}
\end{align}

Simplificando:

\begin{equation}
\boxed{
\begin{aligned}
\ddot{y}_c = &\; \frac{R\alpha_1(D\sin\theta - y_c\cos\theta)(R\sin\theta - y_c)}{(R\sin\theta - y_c)^2} \\
&\; - \frac{R\omega_1^2(D\cos\theta + y_c\sin\theta)(R\sin\theta - y_c)}{(R\sin\theta - y_c)^2} \\
&\; + \frac{R\omega_1\dot{y}_c[R\sin\theta\cos\theta - y_c\cos\theta - D\sin\theta + y_c\cos\theta]}{(R\sin\theta - y_c)^2}
\end{aligned}
}
\end{equation}

\end{document}
