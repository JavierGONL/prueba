\documentclass[12pt]{article}
% Compilar con LuaLaTeX
\usepackage[spanish, es-nodecimaldot]{babel}
\usepackage[a4paper, margin=2.2cm]{geometry}
\usepackage{amsmath}
\usepackage{unicode-math}
\setmainfont{Latin Modern Roman}
\setmathfont{STIX Two Math}

\usepackage{newunicodechar}

% ==== Definición de símbolos geométricos especiales (flechas de dirección) ====
% Código Unicode U+29A8 a U+29AF
\newcommand{\angulese}{\ensuremath{\symbol{"29A8}}} % ⦨
\newunicodechar{⦨}{\angulese}

\newcommand{\anglees}{\ensuremath{\symbol{"29A9}}} % ⦩
\newunicodechar{⦩}{\anglees}

\newcommand{\anglenesw}{\ensuremath{\symbol{"29AA}}} % ⦪
\newunicodechar{⦪}{\anglenesw}

\newcommand{\anglensw}{\ensuremath{\symbol{"29AB}}} % ⦫
\newunicodechar{⦫}{\anglensw}

\newcommand{\measuredangleleft}{\ensuremath{\symbol{"29AC}}} % ⦬
\newunicodechar{⦬}{\measuredangleleft}

\newcommand{\measuredangleright}{\ensuremath{\symbol{"29AD}}} % ⦭
\newunicodechar{⦭}{\measuredangleright}

\newcommand{\sphericalangleup}{\ensuremath{\symbol{"29AE}}} % ⦮
\newunicodechar{⦮}{\sphericalangleup}

\newcommand{\sphericalangledown}{\ensuremath{\symbol{"29AF}}} % ⦯
\newunicodechar{⦯}{\sphericalangledown}

\title{Cómo obtener $\omega_3$ y $\alpha_3$}
\author{}
\date{}

\begin{document}
\maketitle

Este apunte resume, paso a paso, cómo calcular primero la velocidad angular $\omega_3$ y luego la aceleración angular $\alpha_3$ del eslabón $K$ (ángulo $\beta$), a partir del giro del manivela $R$ (ángulo $\theta$) con $\omega_1$ y $\alpha_1$. Se siguen exactamente las ecuaciones de tus notas.

\section*{Geometría}
De la geometría del mecanismo:
\begin{equation}
	\sin\beta \,=\, \frac{R\cos\theta - D}{K}, \qquad
	\beta \,=\, \arcsin\!\left(\frac{R\cos\theta - D}{K}\right),
\end{equation}
tomando la rama que cumpla $\cos\beta>0$ para la configuración mostrada.

\section*{Velocidades}
La velocidad del punto $C$ se descompone como:
\begin{equation}
	\vec v_C \;=\; \vec v_B \;+\; \vec v_{C/B}
\end{equation}
donde:
\begin{itemize}
	\item $\vec v_B = \omega_1\,R$ ⦭ $\theta$
	\item $\vec v_{C/B} = \omega_3\,K$ ⦪ $\beta$
\end{itemize}

Descomponiendo en $x$ y $y$ se obtiene:
\begin{align}
	\text{En }y:\quad v_C \;&=\; \omega_1\,R\cos\theta \; - \; \omega_3\,K\sin\beta, \\
	\text{En }x:\quad 0 \;&=\; \omega_1\,R\sin\theta \; - \; \omega_3\,K\cos\beta.
\end{align}
De la segunda ecuación resulta directamente
\begin{equation}
	\boxed{\;\displaystyle \omega_3 \,=\, \frac{\omega_1\,R\sin\theta}{K\cos\beta}\;}
\end{equation}

\section*{Aceleraciones}
La aceleración del punto $C$ se descompone como:
\begin{equation}
	\vec a_C \;=\; \vec a_B \;+\; \vec a_{C/B}
\end{equation}
donde cada término se compone de componentes normal y tangencial:
\begin{align*}
	\vec a_B &= (a_B)_n + (a_B)_t \\
	&= \omega_1^2 R  ⦫ \theta  +  \alpha_1 R  ⦭\theta \\[1mm]
	\vec a_{C/B} &= (a_{C/B})_n + (a_{C/B})_t \\
	&= \omega_3^2 K ⦬ \beta +  \alpha_3 K ⦪\beta
\end{align*}

Descomponiendo en $x$ y $y$:
\begin{align}
	\text{En }y:\quad a_C \;&=\; -\,\omega_1^{2} R\sin\theta \; + \; \alpha_1 R\cos\theta \; + \; \omega_3^{2} K\cos\beta \; - \; \alpha_3 K\sin\beta, \\[2mm]
	\text{En }x:\quad 0 \;&=\; \omega_1^{2} R\cos\theta \; + \; \alpha_1 R\sin\theta \; - \; \omega_3^{2} K\sin\beta \; - \; \alpha_3 K\cos\beta.
\end{align}
De la ecuación en $x$ se despeja $\alpha_3$ sin necesidad de conocer $a_C$:
\begin{equation}
	\boxed{\;\displaystyle
	\alpha_3 \,=\, \frac{\omega_1^{2} R\cos\theta + \alpha_1 R\sin\theta - \omega_3^{2} K\sin\beta}{K\cos\beta}\;}
\end{equation}
Una vez conocida $\alpha_3$, la componente vertical de la aceleración del punto $C$ es
\begin{equation}
	\boxed{\;\displaystyle a_C \,=\, -\,\omega_1^{2} R\sin\theta + \alpha_1 R\cos\theta + \omega_3^{2} K\cos\beta - \alpha_3 K\sin\beta\;}
\end{equation}

\paragraph{Observación.} Si se desea una expresión totalmente en términos de $R, K, D, \theta, \omega_1, \alpha_1$, basta con sustituir $\beta$ desde la geometría y
\(\omega_3 = \omega_1 R\sin\theta/(K\cos\beta)\) en la fórmula de $\alpha_3$.

\end{document}
