% !TeX program = lualatex
\documentclass{article}

\usepackage[spanish]{babel}
\usepackage{fontspec}
\usepackage{unicode-math}
\usepackage{geometry}
\geometry{a4paper, margin=2.5cm}
\usepackage{amsmath}

% Fuentes
\setmainfont{Latin Modern Roman}
% Usa STIX Two Math para mejor cobertura de símbolos
\setmathfont{STIXTwoMath-Regular.otf}[
  Path = C:/Users/lunit/AppData/Local/Programs/MiKTeX/fonts/opentype/public/stix2-otf/,
  BoldFont = STIXTwoMath-Regular.otf
]

% Soporte de caracteres Unicode especiales usados en el texto
\usepackage{newunicodechar}
\newcommand{\angulese}{\ensuremath{\symbol{"29A8}}} % ⦨
\newunicodechar{⦨}{\angulese}
\newcommand{\anglensw}{\ensuremath{\symbol{"29AB}}} % ⦫
\newunicodechar{⦫}{\anglensw}
\newcommand{\measuredangleright}{\ensuremath{\symbol{"29AD}}} % ⦭
\newunicodechar{⦭}{\measuredangleright}
% Símbolo solicitado: ⦮ (spherical angle up)
\newcommand{\sphericalangleup}{\ensuremath{\symbol{"29AE}}} % ⦮
\newunicodechar{⦮}{\sphericalangleup}

\begin{document}

\subsubsection{Análisis de aceleraciones}

Tal como en el análisis de velocidades sabemos que la barra $O_2A$ tiene velocidad y aceleración angulares, y al no presentar traslación, aplicamos rotación respecto a eje fijo:
\[
  \vec a_a = (\alpha_2 \times \vec r) ,\; ⦭\, \Phi^\circ  + (\boldsymbol{\omega}_2 \times(\boldsymbol{\omega}_2 \times \vec r)) ,\; ⦫\, \Phi^\circ
\] 
Donde el primer y segundo termino representan la aceleracion tangencial y normal de a respectivamente. Ahora podemos reescribir la aceleracion de a como parte de $O_1B$, obtieniendo la siguiente relacion:
\[
  \vec a_a = \vec a'_a + \vec a_{a/f} + \vec a_{cor}
\] 
Donde:
\begin{itemize}
    \item \[
  \vec a'_a =  (\alpha_1 \times \vec L) ,\; ⦫\, \theta^\circ  + (\boldsymbol{\omega}_1 \times(\boldsymbol{\omega}_1 \times \vec L)) ,\; ⦭\, \theta^\circ 
\] 
    \item \[
  \vec a_{cor} =  (2\boldsymbol{\omega}_1 \times \vec V_{a/f}) ,\; ⦮\, \theta^\circ \] 
    \item \[a_{a/f} =  \; ⦫\,\theta^\circ \]
\end{itemize}
Sabiendo esto proyectamos todos los vectores respecto a sus componentes rectangulares, obteniendo respectivamente para X e Y:
\begin{align*}
(a'_{a})_{t} \sin\theta + (a_{a/f}) \cos\theta 
&= \omega_2^2 r \cos(\Phi) + \alpha_2 r \sin(\Phi) 
+ 2\omega_1 V_{a/f} \sin\theta - \omega_1^2 L \cos\theta \\[8pt]
(a'_{a})_{t} \cos\theta - (a_{a/f}) \sin\theta 
&= \alpha_2 r \cos(\Phi) - \omega_2^2 r \sin(\Phi)
+ 2\omega_1 V_{a/f} \cos\theta + \omega_1^2 L \sin\theta
\end{align*}
Solucionando el sistema para $\alpha_1$ y $a_{a/f}$:

\begin{equation}
\alpha_1 = \frac{\omega_2^2 r\sin(\theta-\Phi)+\alpha_2 r\cos(\theta-\Phi)+2\omega_1V_{a/f}}{L}
\end{equation}

\begin{equation}
a_{a/f} = \omega_2^2 r\cos(\Phi-\theta) + {\alpha}_2 r \sin(\Phi-\theta) - \omega_1^2 L
\end{equation}

\end{document}
