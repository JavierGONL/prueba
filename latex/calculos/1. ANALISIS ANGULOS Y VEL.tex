% !TeX program = lualatex
\documentclass{article}

\usepackage[spanish]{babel}
\usepackage{fontspec}
\usepackage{unicode-math}
\usepackage{graphicx}
\usepackage{geometry}
\geometry{a4paper, margin=2.5cm}
\usepackage{amsmath}

\setmainfont{Latin Modern Roman}

% Usa STIX Two Math que tiene los símbolos geométricos
\setmathfont{STIXTwoMath-Regular.otf}[
  Path = C:/Users/lunit/AppData/Local/Programs/MiKTeX/fonts/opentype/public/stix2-otf/,
  BoldFont = STIXTwoMath-Regular.otf
]

\usepackage{newunicodechar}

% ==== Definición de símbolos geométricos especiales ====
\newcommand{\measuredangleright}{\ensuremath{\symbol{"29AD}}} % ⦭
\newunicodechar{⦭}{\measuredangleright}
% =======================================================

\begin{document}

\section{Modelos teóricos empleados}

\subsection{Modelo cinemático}

\subsubsection{Relación entre $\theta$ y $\phi$}

El análisis geométrico del mecanismo permite establecer la relación fundamental entre los ángulos.



A partir del triángulo formado por las longitudes $r$, $d$ y $L$, se obtiene:

(ACA VA EL DIAGRAMA!!!!!)

\begin{equation}
L^2 = r^2 + d^2 - 2rd \cos(180^\circ - \phi)
\end{equation}

sabiendo que  $\cos(180^\circ - \Phi) = -\cos(\phi)$, se tiene:

\begin{equation}
L = \sqrt{r^2 + d^2 + 2rd \cos(\phi)}
\end{equation}

Por la ley de senos:

\begin{equation}
\frac{\sin(180^\circ - \Phi)}{L} = \frac{\sin(\Theta)}{r}
\end{equation}

sabiendo que $\sin(180-\phi) = sin(\phi) $ y despejando:

\begin{equation}
\sin(\theta) = \frac{\sin(\phi) L}{r}
\end{equation}


Sustituyendo la expresión de $L$:

\begin{equation}
\boxed{
\sin(\Theta) =
\frac{\sqrt{r^2 + d^2 + 2rd \cos(\Phi)} \sin(\Phi)}
{r}
}
\end{equation}

obtenemos una Relacion entre $\theta$ y $\phi$.

\subsubsection{Análisis de velocidades}

Sabemos que la barra $O_2A$ tiene velocidad y aceleración angulares, y al no presentar traslación, aplicamos rotación respecto a eje fijo:
\[
  \vec V_a = \boldsymbol{\omega}_2 \times \vec r,\; ⦭\, \Phi^\circ
\]
Podemos escribir una descomposición relativa de $ \vec V_a $:
\[
  \vec V_a = \vec V'_a + \vec V_{a/f}
\]

% Imagen comentada porque el archivo no existe
% \begin{figure}[htbp]
% \centering
% \includegraphics[width=0.27\textwidth]{imagenes/poligono_velocidades.png}
% \caption{Polígono vectorial de velocidades.}
% \label{fig:poligono_velocidades}
% \end{figure}

donde $\vec V'_a$ sigue la trayectoria del sistema de rotación y $\vec V_{a/f}$ es la velocidad relativa.

\begin{equation}
    V_{a/f} = ( \omega_2\cdot r \cdot \sin(\phi - \theta))
\end{equation}

Además:
\begin{equation}
    V_a' = \cos(\Phi -\Theta) \, V_a
\end{equation}

De donde:
\[
    \omega_1 L = [\cos(\Phi -\Theta)]\, \omega_2 r
\]

Por tanto, la velocidad angular del eslabón es:
\begin{equation}
    \boxed{
    \omega_1 = \frac{\omega_2 r}{L} \, \cos(\Phi -\Theta)
    }
\end{equation}

\end{document}