% !TeX program = lualatex
\documentclass{article}

\usepackage[spanish]{babel}
\usepackage{fontspec}
\usepackage{unicode-math}
\usepackage{graphicx}
\usepackage{float} % Para posicionamiento exacto de figuras con [H]
\usepackage{geometry}
\geometry{a4paper, margin=2.5cm}
\usepackage{ragged2e} % Para el entorno justify
\usepackage{abstract} % Para el entorno abstract

\renewcommand{\contentsname}{Contenido}
\renewcommand{\abstractname}{Resumen}

\setmainfont{Latin Modern Roman}

% Usa STIX Two Math que tiene los símbolos geométricos
\setmathfont{STIXTwoMath-Regular.otf}[
  Path = C:/Users/lunit/AppData/Local/Programs/MiKTeX/fonts/opentype/public/stix2-otf/,
  BoldFont = STIXTwoMath-Regular.otf
]

\usepackage{newunicodechar}

% ==== Definición de símbolos geométricos especiales ====
% Código Unicode U+29A8 a U+29AF
\newcommand{\angulese}{\ensuremath{\symbol{"29A8}}} % ⦨
\newunicodechar{⦨}{\angulese}

\newcommand{\anglees}{\ensuremath{\symbol{"29A9}}} % ⦩
\newunicodechar{⦩}{\anglees}

\newcommand{\anglenesw}{\ensuremath{\symbol{"29AA}}} % ⦪
\newunicodechar{⦪}{\anglenesw}

\newcommand{\anglensw}{\ensuremath{\symbol{"29AB}}} % ⦫
\newunicodechar{⦫}{\anglensw}

\newcommand{\measuredangleleft}{\ensuremath{\symbol{"29AC}}} % ⦬
\newunicodechar{⦬}{\measuredangleleft}

\newcommand{\measuredangleright}{\ensuremath{\symbol{"29AD}}} % ⦭
\newunicodechar{⦭}{\measuredangleright}

\newcommand{\sphericalangleup}{\ensuremath{\symbol{"29AE}}} % ⦮
\newunicodechar{⦮}{\sphericalangleup}

\newcommand{\sphericalangledown}{\ensuremath{\symbol{"29AF}}} % ⦯
\newunicodechar{⦯}{\sphericalangledown}
% =======================================================

\begin{document}
{\LARGE\bfseries Diseño conceptual y análisis dinámico de un sistema de martillo accionado por un mecanismo de retorno rápido \par}


% ===== MODELOS TEÓRICOS =====

\section*{Parámetros geométricos y operativos}

\begin{figure}[H]
    \centering
    \includegraphics[width=0.6\textwidth]{{imagenes/diagrama mecanismo}.png}
    \caption{Diagrama geométrico del mecanismo con referencias de dimensiones (r, d, R, K, D).}
    \label{fig:diagrama_mecanismo}
\end{figure}

\vspace{0.5em}
 
\addcontentsline{toc}{subsection}{Parámetros geométricos y operativos (SI)}
Los valores usados en el análisis corresponden al diseño actual del prototipo y se expresan en unidades SI:
\begin{align*}
 r &= 0.070\;\text{m} & d &= 0.110\;\text{m} & R &= 0.200\;\text{m} & \delta = 0.00665\\
 K &= 0.070\;\text{m} & D &= 0.17 \;\text{m} & \lambda &= 0.1093458\;\text{m} & g &= 9.81\;\text{m/s}^2 \\
 \omega_2 &= 4.00\;\text{rad/s} & \alpha_2 &= 0.00\;\text{rad/s}^2
\end{align*}
Se asume motor de velocidad angular constante ($\alpha_2=0$) y todos los ángulos en radianes. El valor de $\lambda$ corresponde a la distancia del pivote $O_1$ al centroide del eslabón $O_1B$.


% ============================================================
% SECCIÓN 2: ANÁLISIS DE ACELERACIONES
% ============================================================
\section{Aceleraciones angulares y relativas}

 Ligadura cinemática para la aceleracion de $a$ respecto a $O_2$:
\[
  \vec a_a = (\alpha_2 \times \vec r) ,\; ⦭\, \Phi^\circ  + (\boldsymbol{\omega}_2 \times(\boldsymbol{\omega}_2 \times \vec r)) ,\; ⦫\, \Phi^\circ
\] 
 Planteamiento respecto a $O_1$:
\[
  \vec a_a = \vec a'_a + \vec a_{a/f} + \vec a_{cor}
\] 
Donde:

\begin{itemize}
    \item \[
  \vec a'_a =  (\alpha_1 \times \vec L) ,\; ⦫\, \theta^\circ  + (\boldsymbol{\omega}_1 \times(\boldsymbol{\omega}_1 \times \vec L)) ,\; ⦭\, \theta^\circ 
\] 
$\vec a'_a$ representa  la aceleracion producto de la rotacion del sistema
    \item \[
  \vec a_{cor} =  (2\boldsymbol{\omega}_1 \times \vec V_{a/f}) ,\; ⦮\, \theta^\circ \] 
$\vec a_{cor}$ representa la aceleracion de coriolis, un termino adicional que surge en sistemas no inerciales de este tipo
    \item \[a_{a/f} =  \; ⦫\,\theta^\circ \]
$a_{a/f}$ representa la aceleracion del punto $a$ relativo al sistema en rotacion
\end{itemize}

% ============================================================
% SECCIÓN 3: LAZO VECTORIAL PARA OBTENER VEL Y ACEL DEL MARTILLO
% ============================================================
\section{Derivacion explicita Lazo vectorial del martillo}

Planteando el lazo vectorial para el mecanismo, se tiene:

% Lazo vectorial (ya existe la imagen en la carpeta 'imagenes')
\begin{figure}[H]
    \centering
    \includegraphics[width=0.6\textwidth]{{imagenes/Lazo vectorial}.png}
    \caption{Lazo vectorial del mecanismo que relaciona $O_1B$, el eslabón $BC$ y el desplazamiento vertical del martillo.}
    \label{fig:lazo_vectorial}
\end{figure}

Al proyectar en los ejes $x$ y $y$ y organizar en una sola ecuacion tenemos:
\begin{equation*}
    K^2 = {(R\cos\theta - D)}^{2} + {(R\sin\theta - y_c)}^{2}
\end{equation*}

\subsection{Derivación explicita para la velocidad}

Derivando ambos lados respecto al tiempo. Donde:
\begin{itemize}
    \item $K$ es constante
    \item $\frac{d\theta}{dt} = \omega_1$
    \item $y_c$ es variable
    \item $R$ es constante
    \item $D$ es constante
\end{itemize}

\begin{align}
\frac{d(K^2)}{dt} &= \frac{d}{dt}\left[(R\cos(\theta) - D)^2 + (R\sin(\theta) - y_c)^2\right]
\end{align}

Como $K$ es constante:
\begin{align}
0 &= \frac{d}{dt}\left[(R\cos(\theta) - D)^2 + (R\sin(\theta) - y_c)^2\right]
\end{align}

Aplicando la regla de la cadena:
\begin{align}
0 &= 2(R\cos(\theta) - D)\frac{d}{dt}(R\cos(\theta) - D) + 2(R\sin(\theta) - y_c)\frac{d}{dt}(R\sin(\theta) - y_c)
\end{align}

Calculando las derivadas internas:
\begin{align}
\frac{d}{dt}(R\cos(\theta) - D) &= -R\sin(\theta)\frac{d\theta}{dt} = -R\sin(\theta)\omega_1 \\
\frac{d}{dt}(R\sin(\theta) - y_c) &= R\cos(\theta)\frac{d\theta}{dt} - \dot{y}_c = R\cos(\theta)\omega_1 - \dot{y}_c
\end{align}

Sustituyendo:
\begin{align}
0 &= 2(R\cos(\theta) - D)(-R\sin(\theta)\omega_1) + 2(R\sin(\theta) - y_c)(R\cos(\theta)\omega_1 - \dot{y}_c)
\end{align}

Dividiendo entre 2:
\begin{align}
0 &= -(R\cos(\theta) - D)R\sin(\theta)\omega_1 + (R\sin(\theta) - y_c)(R\cos(\theta)\omega_1 - \dot{y}_c)
\end{align}

Expandiendo:
\begin{align}
0 &= -R^2\cos(\theta)\sin(\theta)\omega_1 + DR\sin(\theta)\omega_1 \nonumber \\
  &\quad + R^2\sin(\theta)\cos(\theta)\omega_1 - R\sin(\theta)\dot{y}_c \nonumber \\
  &\quad - y_cR\cos(\theta)\omega_1 + y_c\dot{y}_c
\end{align}

Los términos $R^2\cos(\theta)\sin(\theta)\omega_1$ se cancelan, quedando:
\begin{align}
0 &= DR\sin(\theta)\omega_1 - R\sin(\theta)\dot{y}_c - y_cR\cos(\theta)\omega_1 + y_c\dot{y}_c
\end{align}

Reorganizando:
\begin{align}
\dot{y}_c(R\sin(\theta) - y_c) &= R\omega_1(D\sin(\theta) - y_c\cos(\theta))
\end{align}

Dividiendo ambos lados por $(R\sin(\theta) - y_c)$:
\begin{align}
\boxed{\dot{y}_c = \frac{R\omega_1(D\sin(\theta) - y_c\cos(\theta))}{R\sin(\theta) - y_c}}
\end{align}

Esta expresión es válida siempre que $R\sin(\theta) \neq y_c$ (es decir, que el denominador no sea cero).

\subsection{Derivacion explicita para la Aceleración }

Derivando nuevamente la ecuación de velocidad para obtener la aceleración:

\begin{equation}
\ddot{y}_c = 
R \; 
\frac{
\big[ \dot{\omega}_1 A - \omega_1^2 B - \omega_1 \dot{y}_c \cos\theta \big] D_{en}
- \omega_1 A (\dot{y}_c + R \omega_1 \cos\theta)
}{
D_{en}^2
}
\end{equation}

\noindent donde:
\begin{align*}
A &= D\sin\theta - y_c\cos\theta,\\
B &= D\cos\theta + y_c\sin\theta,\\
D_{en} &= R\sin\theta - y_c.
\end{align*}

Nota: $\dot{\omega}_1 = \ddot{\theta} = \alpha_1$ es la aceleración angular.
\\

Expandiendo la expresión anterior:

\begin{equation}
\begin{aligned}
\ddot{y}_c = &\; \frac{R\alpha_1(D\sin\theta - y_c\cos\theta)(R\sin\theta - y_c)}{(R\sin\theta - y_c)^2} \\
&\; - \frac{R\omega_1^2(D\cos\theta + y_c\sin\theta)(R\sin\theta - y_c)}{(R\sin\theta - y_c)^2} \\
&\; + \frac{R\omega_1\dot{y}_c[R\sin\theta\cos\theta - y_c\cos\theta - D\sin\theta + y_c\cos\theta]}{(R\sin\theta - y_c)^2}
\end{aligned}
\end{equation}

Simplificando:
\begin{equation}
\begin{split}
\ddot{y}_c = &\; \frac{R\alpha_1(D\sin\theta - y_c\cos\theta)(R\sin\theta - y_c)}{(R\sin\theta - y_c)^2} \\
&\; - \frac{R\omega_1^2(D\cos\theta + y_c\sin\theta)(R\sin\theta - y_c)}{(R\sin\theta - y_c)^2} \\
&\; + \frac{R\omega_1\dot{y}_c\big(R\sin\theta\cos\theta - D\sin\theta\big)}{(R\sin\theta - y_c)^2}
\end{split}
\end{equation}

% ============================================================
% SECCIÓN 4: CÁLCULO DE ω₃ Y α₃
% ============================================================
\section{Cinemática del eslabón $BC$}
\subsubsection{Ecuacion vectorial de velocidad en el punto $B$}

Usando movimiento del plano general podemos expresar la velocidad en el punto $C$ por medio de la siguiente expresion:
\begin{equation}
	\vec v_C = \vec v_B + \vec v_{C/B}
\end{equation}
En la cual:

\begin{itemize}
	\item $\vec v_B = \omega_1\,R$ ⦭ $\theta$, representa la velocidad del punto $B$ y es producto a la rotacion del eslabón $O_1B$ 
	\item $\vec v_{C/B} = \omega_3\,K$ ⦪ $\beta$, representa la velocidad relativa tangecial del punto $C$ respecto a $B$ y es producto de la rotacion del eslabón $BC$
	\item $\vec v_C = \dot{y}_c$ (dirección vertical ↓), representa la velocidad del martillo, obtenida mediante el metodo de lazo vectorial
\end{itemize}
Tomando en cuenta las inclinaciones proyectamos los vectores en el eje horizontal, obtienendo asi la siguiente relacion para $\omega_3$ y $\omega_1$:
\begin{equation}
	0 = -\omega_1\,R\,\sin\theta - \omega_3\,K\,\cos\beta
\end{equation}

Despejando $\omega_3$ de la ecuacion anterior obtenemos:
\begin{equation}
	\boxed{
		\omega_3 = -\,\frac{\omega_1\,R\,\sin\theta}{K\,\cos\beta}
	}
\end{equation}


\subsection{Ecuacion vectorial de aceleración en el punto $B$}

De manera analoga al las velocidad se tiene un movimiento de plano general, el cual nos permite expresar la aceleracion en el punto $C$ por medio de la siguiente expresion:
\begin{equation}
	\vec a_C = \vec a_B + \vec a_{C/B}
\end{equation}
En el cual:
\begin{itemize}
	\item $\vec a_B = (\vec a_B)_t + (\vec a_B)_n$:
	\begin{itemize}
		\item $(\vec a_B)_t = \alpha_1 \cdot R$, ⦭ $\theta$ , Representa la aceleracion tangecial del punto $B$ debido a la rotacion fija en $O_1$
		\item $(\vec a_B)_n= \omega_1^2 \cdot R$ ⦫ $\theta$, Representa la aceleracion normal producto de la rotacion respecto a $O_1$. Apunta directamente hacia el eje de rotacion
	\end{itemize}
	\item $\vec a_{C/B} = (\vec a_{C/B})_t + (\vec a_{C/B})_n$ en donde:
	\begin{itemize}
		\item $(\vec a_{C/B})_t = \alpha_3 \cdot K$ ⦪ $\beta$ Representa la aceleracion tangecial relativa del punto $C$ respecto a $B$, es producto de la rotacion del eslabon $BC$
		\item $(\vec a_{C/B})_n = \omega_3^2 \cdot K$ ⦯ $\beta$ Representa la aceleracion normal relativa del punto $C$ respecto a $B$, apunta directamente hacia el eje de rotacion del eslabon $BC$
	\end{itemize}
	\item $\vec a_C = \ddot{y}_c$ ↓ Representa la aceleracion del martillo, obtenida mediante el metodo de lazo vectorial
\end{itemize}
Con esta ecuacion y por medio de las proyecciones en el eje horizontal de cada vector encontramos la siguiente relacion que describe a $\alpha_3$ en funcion de $\alpha_1$ y $\omega_1$:
\begin{equation}
	0 = -\alpha_1 R\sin\theta + \omega_1^2 R\cos\theta
	    + \alpha_3 K\cos\beta - \omega_3^2 K\sin\beta
\end{equation}

Despejando $\alpha_3$:
\begin{equation}
	\boxed{
		\alpha_3 = \frac{\omega_1^2 R\cos\theta + \alpha_1 R\sin\theta - \omega_3^2 K\sin\beta}{K\cos\beta}
	}
\end{equation}

% ============================================================
% SECCIÓN 5: ANÁLISIS CINÉTICO
% ============================================================

\section{Análisis cinético}
Las ecuaciones de movimiento se obtuvieron mediante los siguientes diagramas de cuepro libre: 
\begin{figure}[H]
    \centering
    \includegraphics[width=0.6\textwidth]{{imagenes/diagrama C}.png}
    \caption{Diagrama C.}
    \label{fig:diagrama_C}
\end{figure}

\begin{figure}[H]
    \centering
    \includegraphics[width=0.6\textwidth]{{imagenes/diagrama BC}.png}
    \caption{Diagrama BC.}
    \label{fig:diagrama_BC}
\end{figure}

\begin{figure}[H]
    \centering
    \includegraphics[width=0.6\textwidth]{{imagenes/diagrama O1B}.png}
    \caption{Diagrama O1B.}
    \label{fig:diagrama_O1B}
\end{figure}

\begin{figure}[H]
    \centering
    \includegraphics[width=0.6\textwidth]{{imagenes/diagrama O2A}.png}
    \caption{Diagrama O2A.}
    \label{fig:diagrama_O2A}
\end{figure}
---

\section{Notas finales}
La resolución numérica se implementa en Python generando curvas de $\theta(t)$, $\omega_1(t)$, $\dot y_c(t)$, $\ddot y_c(t)$, $\beta(t)$, $\omega_3(t)$, $\alpha_3(t)$, fuerzas y par $\tau(t)$, además de una búsqueda del parámetro $D$ que maximiza la carrera evitando singularidades (denominadores cercanos a cero en la expresión de $\dot y_c$). Esto garantiza un régimen estable y aprovechamiento máximo del desplazamiento del martillo.
\section{Fase de Diseño Detallado}

\section{Análisis de los resultados teóricos}

\section{Análisis sobre el funcionamiento del prototipo}

\section{Conclusiones y recomendaciones}

% ===== REFERENCIAS =====

\addcontentsline{toc}{section}{Referencias}
\begin{thebibliography}{99}


\bibitem{beer}
Ferdinand P. Beer, E. Russell Johnston Jr., William E. Clausen, "Dinámica para ingeniería", McGraw-Hill, última edición disponible.

\bibitem{hal-01715664}
Autor desconocido, “Vector loop approach for mechanism analysis,” HAL Open Science, 2018. Disponible en: https://hal.science/hal-01715664/document

\bibitem{stix2}
STIX Fonts Project, “STIX Two Math,” 2020. Disponible en: https://stixfonts.org

\end{thebibliography}

\end{document}
