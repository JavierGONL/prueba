% !TeX program = lualatex
\documentclass{article}

\usepackage[spanish]{babel}
\usepackage{fontspec}
\usepackage{unicode-math}
\usepackage{graphicx}
\usepackage{float} % Para posicionamiento exacto de figuras con [H]
\usepackage{microtype} % Microtipografía: mejor espaciado y cortes
\usepackage{caption} % Control de espaciado en captions
\usepackage{subcaption} % Para subfiguras
\usepackage{geometry}
\geometry{a4paper, margin=2.5cm}
\usepackage{ragged2e} % Para el entorno justify
\usepackage{abstract} % Para el entorno abstract
\usepackage[hidelinks]{hyperref} % Para enlaces clickeables sin cuadros de colores

\renewcommand{\contentsname}{Contenido}
\renewcommand{\abstractname}{Resumen}

% Ajustes globales de espaciado alrededor de figuras y captions
\setlength{\intextsep}{8pt plus 2pt minus 2pt}
\setlength{\textfloatsep}{8pt plus 2pt minus 2pt}
\setlength{\floatsep}{8pt plus 2pt minus 2pt}
\captionsetup{aboveskip=6pt,belowskip=6pt}

\setmainfont{Latin Modern Roman}

% Usa STIX Two Math que tiene los símbolos geométricos
\setmathfont{STIXTwoMath-Regular.otf}[
  Path = C:/Users/lunit/AppData/Local/Programs/MiKTeX/fonts/opentype/public/stix2-otf/,
  BoldFont = STIXTwoMath-Regular.otf
]

\usepackage{newunicodechar}

% ==== Definición de símbolos geométricos especiales ====
% Código Unicode U+29A8 a U+29AF
\newcommand{\angulese}{\ensuremath{\symbol{"29A8}}} % ⦨
\newunicodechar{⦨}{\angulese}

\newcommand{\anglees}{\ensuremath{\symbol{"29A9}}} % ⦩
\newunicodechar{⦩}{\anglees}

\newcommand{\anglenesw}{\ensuremath{\symbol{"29AA}}} % ⦪
\newunicodechar{⦪}{\anglenesw}

\newcommand{\anglensw}{\ensuremath{\symbol{"29AB}}} % ⦫
\newunicodechar{⦫}{\anglensw}

\newcommand{\measuredangleleft}{\ensuremath{\symbol{"29AC}}} % ⦬
\newunicodechar{⦬}{\measuredangleleft}

\newcommand{\measuredangleright}{\ensuremath{\symbol{"29AD}}} % ⦭
\newunicodechar{⦭}{\measuredangleright}

\newcommand{\sphericalangleup}{\ensuremath{\symbol{"29AE}}} % ⦮
\newunicodechar{⦮}{\sphericalangleup}

\newcommand{\sphericalangledown}{\ensuremath{\symbol{"29AF}}} % ⦯
\newunicodechar{⦯}{\sphericalangledown}
% =======================================================

\begin{document}

% ===== PORTADA =====
\begin{titlepage}
    \centering
    
    % Logo de la Universidad Nacional de Colombia 
    \includegraphics[width=4.5cm]{imagenes/escudo_unal.png}
    
    \vspace{1.5cm}
    
    {\LARGE\bfseries Diseño Conceptual y Análisis Dinámico de un Sistema de Martillo Accionado por un Mecanismo de Retorno Rápido \par}
    
    \vspace{1.5cm}
    

    \vspace{2cm}
    
    {\large\bfseries Kevin Javier Gonzalez Luna\par}
    {\large  kegonzalezl@unal.edu.co\par}
    
    \vspace{1cm}
    
    {\large\bfseries Ivan Felipe Maluche Suarez \par}
    {\large imaluche @unal.edu.co\par}
    
    \vspace{1cm}
    
    {\large\bfseries Juan David Hernández Daza\par}
    {\large jhernandezda@unal.edu.co\par}

    \vspace{1cm}
    
    {\large\bfseries Joseph Nicolas Mahecha Cruz\par}
    {\large jomahecha@unal.edu.co\par}
    \vfill
    
    {\large Nelson Arzola de la Pe{\~n}a\\
    \vspace{1cm}
    Universidad Nacional de Colombia,\\
    Bogotá D.C.\\
    Facultad de Ingeniería.\\
    2025.\par}
    
\end{titlepage}

% ===== INICIO DEL CONTENIDO =====
\newpage

% ===== RESUMEN =====
\addcontentsline{toc}{section}{Resumen}

\begin{abstract}
El proyecto presenta el diseño y el modelado cinemático de un mecanismo de retorno rápido tipo Whitworth, modificado para funcionar como un martillo de impacto vertical. Su propósito es convertir el movimiento rotacional uniforme entregado por un motor el\'ectrico con reductor en un movimiento alternativo de avance y retroceso con velocidades angulares efectivas distintas, de modo que el percutor descienda lentamente y retorne con rapidez.


En la propuesta de redise\~no, el rodillo deslizante cl\'asico se sustituye por un tornillo pasante que transmite directamente el esfuerzo entre la manivela motriz y la palanca ranurada; esta modificaci\'on reduce el n\'umero de componentes, simplifica la manufactura y conserva la eficiencia cinem\'atica del conjunto.


La transmisi\'on de energ\'ia se realiza desde el eje de entrada $O_{2}$ hasta la palanca oscilante pivotada en $O_{1}$, la cual acciona una biela articulada que mueve un bloque deslizante (martillo) fabricado en acero A36 y guiado por dos varillas lineales, generando un golpe controlado durante la fase de avance.


El mecanismo est\'a contenido en un volumen m\'aximo de $60\times60\times15\ \mathrm{cm}$ y se anal\'izo mediante el m\'etodo del lazo vectorial, determinando relaciones angulares as\'i como velocidades y aceleraciones instant\'aneas. Los resultados muestran una correlaci\'on directa entre la velocidad angular de entrada y la aceleraci\'on lineal del martillo, corroborando el comportamiento asim\'etrico de retorno r\'apido.


En conjunto, el dise\~no constituye una alternativa compacta, funcional y did\'actica para el estudio pr\'actico de la conversi\'on de movimiento rotacional en movimiento lineal alternativo, con aplicaciones tanto en entornos acad\'emicos como en prototipos mec\'anicos ligeros.
\end{abstract}


\noindent\textbf{Palabras clave:} mecanismo de retorno rápido de Whitworth; Cinemática, Cinética, Impacto, Desplazamiento, Mecanismo, Martillo.
\vspace{0.5cm}

\newpage
% ===== TABLA DE CONTENIDO =====
\tableofcontents
\newpage


% ===== INTRODUCCIÓN =====
\section{Introducción}

\begin{justify}
El presente informe aborda el diseño, modelado y verificación de un mecanismo de retorno rápido aplicado a un martillo de impacto vertical. Motivado por la necesidad de convertir un giro rotacional continuo en un desplazamiento alternado con fases asimétricas de trabajo y recuperación, el proyecto propondra una adaptación del esquema Whitworth —reemplazando el rodillo por un tornillo pasante— para reducir componentes, mejorar la rigidez y facilitar la fabricación. La metodología integra formulación geométrica mediante el lazo vectorial, derivación de la Jacobiana, modelado CAD, simulaciones dinámicas y la construcción experimental de un prototipo para validar recorridos, aceleraciones y pares requeridos. El alcance incluye el dimensionamiento geométrico, análisis cinematográfico y cinético, planos de fabricación y pruebas funcionales; el documento se ordena para presentar la fundamentación teórica, el diseño conceptual y detallado, el modelado, las pruebas y las conclusiones.
\end{justify}

% ===== FUNDAMENTACIÓN TEÓRICA =====
\section{Fundamentación teórica}

El análisis de mecanismos planos requiere el uso de conceptos fundamentales de cinemática y cinética, entre estos para el análisis del mecanismo se harán uso de:

\subsection{Movimiento de plano general}

El movimiento de plano general describe los valores de velocidad y aceleración de un punto de un sólido rígido por medio de las siguientes ecuaciones: 

\begin{equation}
\vec{V}_b = \vec{V}_a + \vec{V}_{b/a} = \vec{V}_a + (\vec{\omega} \times \vec{r}_{b/a})
\end{equation}

\begin{equation}
\vec{a}_b = \vec{a}_a + \vec{a}_{b/a} = \vec{a}_a + (\vec{\alpha} \times \vec{r}_{b/a}) + (\vec{\omega} \times (\vec{\omega} \times \vec{r}_{b/a}))
\end{equation}

Donde: 
\begin{itemize}
    \item $\vec{\omega}$ y $\vec{\alpha}$ representan la velocidad y aceleración angular respectivamente, son iguales a la primera y segunda derivada del angulo de rotacion respecto al tiempo ($\dot{\theta}, \ddot{\theta}$)
    \item $\vec{r}_{b/a}$ representa el vector que une $\vec{r_a}$ con $\vec{r_b}$
\end{itemize}
\subsection{Movimiento plano de una particula relativa a un sistema de referencia en rotacion}
Cuando se presentan casos donde el objeto se mueve dentro de un sistema en rotacion (como es en nuestro caso el pasador que se mueve linealmente sobre la barra $O1B$) es necesario usar las siguiente ecuaciones para encontrar la variables cinematicas de un punto p dado: 

\begin{equation}
\vec{V}_p = \vec{V}'_p + \vec{V}_{p/f} 
\end{equation}

\begin{equation}
\vec{a}_p = \vec{a}'_p + \vec{a}_{p/f} + a_{cor} 
\end{equation}
Donde: 
\begin{itemize}
    \item $\vec{V'_p}$ y $\vec{a'_p}$ representan las variables cinematicas que posee el cuerpo producto de la rotacion del sistema de referencia, se puede entender como un punto del sistema de referencia que coincide con p en todo momento
    \item $\vec{V}_{p/f}$ y $\vec{a}_{p/f}$ representan las variables cinematicas del cuerpo relativas al sistema, en nuestro caso al ser un pasador sigue la geometria de la barra ${r_b}$
    \item $\vec{a}_{cor}$ representa la aceleracion de coreolis, la cual debe ser igual a $2\vec{\Omega} \times \vec{V_{p/f}}$ y es propia del sistema en rotacion
\end{itemize}

\subsection{Método del lazo vectorial}

Este método analiza mecanismos a partir de ecuaciones vectoriales que cierran un circuito. Cada eslabón se modela como un vector de posición en un sistema de coordenadas global $XY$. Con las longitudes conocidas y los ángulos de entrada, se plantea el lazo de cierre para obtener relaciones geométricas entre las posiciones de los eslabones. Luego, dichas relaciones se derivan una y dos veces respecto al tiempo para obtener, respectivamente, las velocidades y las aceleraciones instantáneas [2].

En la práctica, el procedimiento consiste en:

\begin{itemize}
    \item Plantear el cierre vectorial y su proyección en los ejes $x$ y $y$.
    \item Resolver la geometría para las incógnitas instantáneas (p. ej., $\theta$, $\beta$, $y_c$).
    \item Derivar una vez para obtener las ecuaciones de velocidad y una segunda vez para las de aceleración.
\end{itemize}

\subsection{Ecuaciones de movimiento de un cuerpo rigido}

las ecuaciones de movimiento de un cuerpo rigido me relacionan los vectores de fuerza efectiva con las fuerzas de reaccion del mecanismo:

\begin{itemize}
    \item  $\sum \vec{F} = m \bar{a}$
    \item  $\sum \vec{M_G} = \dot{H_G}$
\end{itemize}

Donde:
\begin{itemize}
    \item $\sum \vec{F}$ es la suma de todas las fuerzas externas aplicadas al cuerpo.
    \item $\sum \vec{M_G}$ es la suma de los momentos respecto al centro de masa $G$.
    \item $\dot{H_G}$ es la derivada temporal del momento angular respecto al centro de masa. 
    \item $\dot{H_G} = \bar{I} \alpha$
\end{itemize}
Estas ecuaciones permiten analizar el movimiento de cuerpos rígidos sometidos a fuerzas y momentos, relacionando las fuerzas efectivas con las reacciones internas del mecanismo. Son la base para el estudio dinámico de sistemas mecánicos como el mecanismo de martillo presentado en este informe.

% ===== FASE DE DISEÑO CONCEPTUAL =====

\section{Fase de Diseño Conceptual}
\subsection{Fase de Diseño Conceptual (versión ampliada con formulación cinemática)}

El mecanismo consiste en un sistema de barras articuladas accionado por un motor que produce el movimiento de martillo. Los componentes principales son: el motor (proporciona rotación constante con velocidad angular $\omega_2$), la barra motriz $(O_2A)$ de longitud $r$ conectada al motor en $O_2$ y al punto $A$, la barra acoplada $(O_1B)$ de longitud aparente variable $L$ articulada en $O_1$ y conectada en $A$, y el sistema de martillo constituido por una biela que transmite el movimiento al bloque deslizante de longitud constante $K$.

Durante la fase conceptual, cada integrante del equipo presentó propuestas de adaptación del mecanismo (variaciones geométricas en bielas, alternativas de guiado, y soluciones híbridas rodillo/tornillo). Se evaluaron múltiples configuraciones bajo criterios de simplicidad constructiva, eficiencia cinemática, robustez frente a impactos y facilidad de fabricación. La solución adoptada —reorientar el plano de movimiento a posición vertical y sustituir el rodillo por un tornillo pasante— fue seleccionada por entregar el mejor compromiso entre reducción de piezas, rigidez y mantenimiento del comportamiento asimétrico característico de retorno rápido.

\subsection{Modelo de prototipo inicial con formulación matemática}

Para el modelado cinemático se emplea el método del lazo vectorial. Definiendo el ángulo de entrada
\[
\delta(t)=\omega_2 t,
\]
y tomando el lazo cerrado entre los puntos característicos del mecanismo, se puede escribir
\[
\overrightarrow{O_2A}(\delta) + \overrightarrow{AB}(\alpha,L) + \overrightarrow{BO_1} = \overrightarrow{O_2O_1},
\]
donde $\alpha$ y $L$ son funciones implícitas de $\delta$ determinadas por la geometría. Si se selecciona un sistema de referencia con $O_2$ en el origen y $\overrightarrow{O_2O_1} = (D,0)$, el lazo se descompone en sus componentes escalares
\begin{align}
r\cos\delta + L(\delta)\cos\alpha(\delta) + b_x &= D,\label{eq:loop_x}\\
r\sin\delta + L(\delta)\sin\alpha(\delta) + b_y &= 0,\label{eq:loop_y}
\end{align}
donde $(b_x,b_y)$ expresa la posición relativa conocida de la porción final del subensamble $(B\!\to\!O_1)$ respecto a $O_1$ (constante según la construcción). Las Eqs. \eqref{eq:loop_x}--\eqref{eq:loop_y} constituyen el sistema no lineal que relaciona $\alpha$ y $L$ con la entrada $\delta$.

Diferenciando temporalmente el lazo se obtienen las relaciones de velocidades:
\[
\frac{d}{dt}\Big(\overrightarrow{O_2A}+\overrightarrow{AB}+\overrightarrow{BO_1}\Big)=\mathbf{0},
\]
que en forma expandida vectorial es
\[
r\dot\delta\begin{pmatrix}-\sin\delta\\[2pt]\cos\delta\end{pmatrix}
+ \dot L\begin{pmatrix}\cos\alpha\\[2pt]\sin\alpha\end{pmatrix}
+ L\dot\alpha\begin{pmatrix}-\sin\alpha\\[2pt]\cos\alpha\end{pmatrix}
= \begin{pmatrix}0\\[2pt]0\end{pmatrix}.
\]
Esta expresión puede reordenarse y, mediante diferenciación implícita de las Eqs. \eqref{eq:loop_x}--\eqref{eq:loop_y}, resolverse numérica o analíticamente (según el caso geométrico) para obtener $\dfrac{d\alpha}{d\delta}$ y $\dfrac{dL}{d\delta}$, y finalmente las velocidades $\dot\alpha$ y $\dot L$ en función de $\dot\delta$.

Para la descripción del movimiento lineal del martillo se define la coordenada de interés $x_E$ (posición del extremo o del centro de masa del bloque deslizante) como función de las variables generalizadas:
\[
x_E = x_E(\delta,\alpha,L).
\]
Aplicando la regla de la cadena, la velocidad y aceleración lineal resultan
\begin{align}
\dot x_E &= \frac{\partial x_E}{\partial\delta}\dot\delta + \frac{\partial x_E}{\partial\alpha}\dot\alpha + \frac{\partial x_E}{\partial L}\dot L
       = J(\delta,\alpha,L)\,\dot{\mathbf{q}},\label{eq:velJ}\\
\ddot x_E &= J(\delta,\alpha,L)\,\ddot{\mathbf{q}} + \dot J(\delta,\alpha,L,\dot{\mathbf{q}})\,\dot{\mathbf{q}},\label{eq:accJ}
\end{align}
donde $\dot{\mathbf{q}}=[\dot\delta,\;\dot\alpha,\;\dot L]^{T}$ y $J$ es la matriz Jacobiana que relaciona velocidades generalizadas con la velocidad lineal del martillo. Si la entrada es \(\delta(t)=\omega_2 t\) con \(\ddot\delta=0\) (velocidad angular constante), la aceleración \(\ddot x_E\) queda dominada por los términos cuadráticos en \(\dot\delta\) (i.e., \(\propto \dot\delta^2\)), lo que evidencia por qué la dinámica del martillo es altamente sensible al valor de \(\omega_2\) y por qué el retorno rápido se manifiesta como un pico de aceleración en la fase de recuperación.

Estas expresiones (lazo vectorial, diferenciación implícita y formulación Jacobiana) son la base para:
\begin{itemize}
  \item implementar el modelo numérico que calcule $\alpha(\delta)$ y $L(\delta)$ para una disposición geométrica dada,
  \item obtener $\dot x_E(\delta)$ y $\ddot x_E(\delta)$ y, a partir de ellas, estimar velocidades de impacto y cargas dinámicas,
  \item estudiar la sensibilidad del sistema frente a variaciones de $\omega_2$ y parámetros geométricos.
\end{itemize}

\subsection{Consideraciones prácticas y de diseño}

Como complementos prácticos se mantienen las recomendaciones ya mencionadas:s repetidos; guías lineales para reducir desviaciones y desgaste; especificación de tolerancias funcionales en articulaciones y alojamientos; y la instrumentación mínima necesaria para validar experimentalmente $\delta(t)$, $\dot x_E(t)$ y $\ddot x_E(t)$ frente a las predicciones teóricas.

El modelo presentado permanece sujeto a ajustes definitivos (disposición final de ejes, anclajes y soportes) que condicionarán las expresiones específicas de $x_E(\cdot)$ y, por tanto, las soluciones cerradas de las ecuaciones mostradas. Si querés, tomo tus dimensiones geométricas y te devuelvo las expresiones cerradas de $\alpha(\delta)$, $L(\delta)$ y las fórmulas explícitas de $x_E(\delta)$, $\dot x_E(\delta)$ y $\ddot x_E(\delta)$ para uso directo en simulación numérica.

% ===== MODELOS TEÓRICOS =====
\section{modelo cinemático}

\subsection*{Parámetros geométricos y operativos}

\begin{figure}[H]
    \centering
    \includegraphics[width=0.45\textwidth]{imagenes/diagrama mecanismo.png}
    \caption{Diagrama geométrico del mecanismo con referencias de dimensiones (r, d, R, K, D).}
    \label{fig:diagrama_mecanismo}
\end{figure}

\vspace{0.5em}
 
Los valores usados en el análisis corresponden al diseño actual del prototipo y se expresan en unidades SI:
\begin{align*}
 r &= 0.070\;\text{m} & d &= 0.110\;\text{m} & R &= 0.200\;\text{m} & \delta = 0.00665\\
 K &= 0.070\;\text{m} & D &= 0.17 \;\text{m} & \lambda &= 0.1093458\;\text{m} & g &= 9.81\;\text{m/s}^2 \\
 \omega_2 &= 4.00\;\text{rad/s} & \alpha_2 &= 0.00\;\text{rad/s}^2
\end{align*}
Se asume motor de velocidad angular constante ($\alpha_2=0$) y todos los ángulos en radianes. El valor de $\lambda$ corresponde a la distancia del pivote $O_1$ al centroide del eslabón $O_1B$.

\subsection[Relación geométrica entre theta y phi]{Relación geométrica entre $\theta$ y $\phi$}

Inicialmente para el analisis cinemático es necesario encontrar expresiones que permitan describrir las distancias
e inclinaciones de todas las partes del mecanismo que conforman el sistema en funcion de una unica variable, para este fin y por medio de un analisis geometrico es posible establecer la siguiente relacion fundamental entre los ángulos por medio del triángulo formado por las longitudes $r$, $d$ y $L$. De esta manera obtenemos:

\begin{equation*}
L^2 = r^2 + d^2 - 2rd \cos(180^\circ - \phi)
\end{equation*}

Por medio de la propiedad trigonometrica  $\cos(180^\circ - \Phi) = -\cos(\phi)$, reducimos la expresion a:

\begin{equation*}
L = \sqrt{r^2 + d^2 + 2rd \cos(\phi)}
\end{equation*}

Donde L representara la distancia desde el punto O hasta el punto A. Sabiendo esto y usando la ley de senos es posible obtener la siguiente expresion:

\begin{equation*}
\frac{\sin(180^\circ - \Phi)}{L} = \frac{\sin(\Theta)}{r}
\end{equation*}

Usando la propiedad trigonometrica $\sin(180-\phi) = sin(\phi) $ y despejando r del lado derecho de la ecuacion obtenemos:

\begin{equation*}
\sin(\theta) = \frac{\sin(\phi) r}{L}
\end{equation*}


Sustituyendo $L$ en la expresion obtenemos:

\begin{equation}
\boxed{
\sin(\Theta) =
\frac{r \sin(\Phi)}
{\sqrt{r^2 + d^2 + 2rd \cos(\Phi)}}
}
\end{equation}

De esta forma se establece una Relacion directa entre el angulo $\theta$ y $\phi$.

\subsection{Velocidades angulares y relativas}

Con la geometria del sistema definimos la velocidad en el punto $a$ por medio de las restricciones cinematicas de cada una de las partes a las que pertenece.
En primera instancia sabemos que el punto $a$ respecto la barra $O_2A$ tiene velocidad y aceleración angulares, mas no presenta traslacion. Por tanto se puede plantear la siguiente ecuacion de rotacion respecto a un eje fijo:
\[
  \vec V_a = \boldsymbol{\omega}_2 \times \vec r,\; ⦭\, \Phi^\circ
\]
De igual manera planteamos al punto $a$ respecto a la barra $O_1B$, en la cual al generar un movimiento relativo a un sistema en rotacion es posible obtener la siguiente ecuacion vectorial para $ \vec V_a $:
\[
  \vec V_a = \vec V'_a + \vec V_{a/f}
\]


donde $\vec V'_a$ representa la velocidad producto de la rotacion del sistema mientras que $\vec V_{a/f}$ representa la velocidad relativa de $a$ en el sistema.

\begin{figure}[H]
    \centering
    \includegraphics[width=0.3\textwidth]{imagenes/Poligono velocidades A.png}
    \caption{Polígono de velocidades en el punto A.}
    \label{fig:poligono_velocidades_A}
\end{figure}

Analizando el diagrama y por medio del teorema de $\sin(\phi - \theta) = \frac{V_{a/f}}{V_a}$ obtenemos:

\begin{equation*}
    V_{a/f} = ( \omega_2\cdot r \cdot \sin(\phi - \theta))
\end{equation*}

Usando $\cos(\phi - \theta) = \frac{V_a'}{V_a}$  tambien obtenemos la siguiente expresion para la velocidad $V'_a$: 
\begin{equation*}
    V_a' = \cos(\Phi -\Theta) \, V_a
\end{equation*}

Finalmente a partir de esta ultima relacion, remplazando con sus valores producto de su rotacion respecto a un eje fijo y despejando para $\omega_1$, obtenemos:

\begin{equation}
    \boxed{
    \omega_1 = \frac{\omega_2 r}{L} \, \cos(\Phi -\Theta)
    }
\end{equation}

% ============================================================
% SECCIÓN 2: ANÁLISIS DE ACELERACIONES
% ============================================================

\subsection{Aceleraciones angulares y relativas}

De manera similar a en el analisis de velocidades tomamos a $a$ respecto a la barra $O_2A$ tiene velocidad y aceleración angular y no presenta traslación, debido a esto aplicamos nuevamente rotación respecto a eje fijo y obtenemos la siguiente expresion vectorial:
\[
  \vec a_a = (\alpha_2 \times \vec r) ,\; ⦭\, \Phi^\circ  + (\boldsymbol{\omega}_2 \times(\boldsymbol{\omega}_2 \times \vec r)) ,\; ⦫\, \Phi^\circ
\] 
Donde el primer termino representa la aceleracion tangencial de $a$ mientras el segundo representa su aceleracion normal.
Hecho esto ahora consideramos a como parte de la barra $O_1B$, obteniendo la sigueinte relacion:
\[
  \vec a_a = \vec a'_a + \vec a_{a/f} + \vec a_{cor}
\] 
Donde:

\begin{itemize}
    \item \[
  \vec a'_a =  (\alpha_1 \times \vec L) ,\; ⦫\, \theta^\circ  + (\boldsymbol{\omega}_1 \times(\boldsymbol{\omega}_1 \times \vec L)) ,\; ⦭\, \theta^\circ 
\] 
    \item \[
  \vec a_{cor} =  (2\boldsymbol{\omega}_1 \times \vec V_{a/f}) ,\; ⦮\, \theta^\circ \] 
    \item \[a_{a/f} =  \; ⦫\,\theta^\circ \]
\end{itemize}

Igualando ambas ecuaciones vectoriales para encontrar las incognitas requeridas proyectamos todos los vectores respecto a sus componentes rectangulares, obteniendo las siguientes ecuaciones para X e Y:
\begin{align*}
(a'_{a})_{t} \sin\theta + (a_{a/f}) \cos\theta 
&= \omega_2^2 r \cos(\Phi) + \alpha_2 r \sin(\Phi) 
+ 2\omega_1 V_{a/f} \sin\theta - \omega_1^2 L \cos\theta \\[8pt]
(a'_{a})_{t} \cos\theta - (a_{a/f}) \sin\theta 
&= \alpha_2 r \cos(\Phi) - \omega_2^2 r \sin(\Phi)
+ 2\omega_1 V_{a/f} \cos\theta + \omega_1^2 L \sin\theta
\end{align*}
Solucionando el sistema para $\alpha_1$ y $a_{a/f}$:

\begin{equation*}
\alpha_1 = \frac{\omega_2^2 r\sin(\theta-\Phi)+\alpha_2 r\cos(\theta-\Phi)+2\omega_1V_{a/f}}{L}
\end{equation*}

\begin{equation*}
a_{a/f} = \omega_2^2 r\cos(\Phi-\theta) + {\alpha}_2 r \sin(\Phi-\theta) - \omega_1^2 L
\end{equation*}

% ============================================================
% SECCIÓN 3: LAZO VECTORIAL PARA OBTENER VEL Y ACEL DEL MARTILLO
% ============================================================
\subsection{Lazo vectorial del martillo: posición, velocidad y aceleración}

Planteando el lazo vectorial para el mecanismo, se tiene:

% Lazo vectorial (ya existe la imagen en la carpeta 'imagenes')
\begin{figure}[H]
    \centering
    \includegraphics[width=0.3\textwidth]{imagenes/Lazo vectorial.png}
    \caption{Lazo vectorial del mecanismo que relaciona $O_1B$, el eslabón $BC$ y el desplazamiento vertical del martillo.}
    \label{fig:lazo_vectorial}
\end{figure}

Al proyectar en los ejes $x$ y $y$:

\begin{align*}
    X: D - R\cos\theta + K\sin\beta &= 0 \\
    Y: y_c + K\cos\beta - R\sin\theta &= 0
\end{align*}

Elevando ambas ecuaciones al cuadrado y sumando:

\begin{equation*}
    K^2 = {(R\cos\theta - D)}^{2} + {(R\sin\theta - y_c)}^{2}
\end{equation*}

\subsubsection{Resultado para la velocidad}

Dividiendo la expresion anterior, obtenemos:

\begin{align}
\boxed{\dot{y}_c = \frac{R\omega_1(D\sin(\theta) - y_c\cos(\theta))}{R\sin(\theta) - y_c}}
\end{align}

Esta expresión es válida siempre que $R\sin(\theta) \neq y_c$ (es decir, que el denominador no sea cero).

\subsubsection{Aceleración mediante lazo vectorial}

Derivando nuevamente la ecuación de velocidad para obtener la aceleración:

\begin{equation}
\ddot{y}_c = 
R \; 
\frac{
\big[ \dot{\omega}_1 A - \omega_1^2 B - \omega_1 \dot{y}_c \cos\theta \big] D_{en}
- \omega_1 A (\dot{y}_c + R \omega_1 \cos\theta)
}{
D_{en}^2
}
\end{equation}

\noindent donde:
\begin{align*}
A &= D\sin\theta - y_c\cos\theta,\\
B &= D\cos\theta + y_c\sin\theta,\\
D_{en} &= R\sin\theta - y_c.
\end{align*}

Nota: $\dot{\omega}_1 = \ddot{\theta} = \alpha_1$ es la aceleración angular.

% ============================================================
% SECCIÓN 4: CÁLCULO DE ω₃ Y α₃
% ============================================================
\section[Cinemática del eslabón BC]{Cinemática del eslabón $BC$}
\subsection[Ángulo beta del eslabón K]{Ángulo $\beta$ del eslabón $K$}

Antes de obtener las variables cineticas del mecanismo es necesario preparar el terreno por medio de algunas relaciones cinematicas adicionales las cuales no fueron necesarias con anterioridad para encontrar los puntos de inetres.

\subsubsection{Inclinacion eslabón K}
Para encontrar el valor del angulo $\beta$ el cual describe la inclinacion del eslabón respecto a la vertical se hace uso de la proyeccion en X desarrollada por medio del lazo vectorial, la cual al despejar $\beta$ nos permite obtener la siguiente expresion:
\begin{equation*}
	\sin\beta = \frac{R\cos\theta - D}{K}, \qquad
	\beta = \arcsin\left(\frac{R\cos\theta - D}{K}\right)
\end{equation*}

\subsubsection{Ecuacion vectorial de velocidad en el punto $B$}

Usando movimiento del plano general podemos expresar la velocidad en el punto $C$ por medio de la siguiente expresion:
\begin{equation*}
	\vec v_C = \vec v_B + \vec v_{C/B}
\end{equation*}
En la cual:

\[
 V_c (↓) = \omega_1 \cdot R (⦭ \theta) + \omega_3 \cdot K (⦪\beta)
\]

Tomando en cuenta las inclinaciones proyectamos los vectores en el eje horizontal, obtienendo asi la siguiente relacion para $\omega_3$ y $\omega_1$:
\begin{equation*}
	0 = -\omega_1\,R\,\sin\theta - \omega_3\,K\,\cos\beta
\end{equation*}

Despejando $\omega_3$ de la ecuacion anterior obtenemos:
\begin{equation}
	\boxed{
		\omega_3 = -\,\frac{\omega_1\,R\,\sin\theta}{K\,\cos\beta}
	}
\end{equation}


\subsubsection{Ecuacion vectorial de aceleración en el punto $B$}

De manera similar a en el caso de las velocidad se tiene un movimiento de plano general, el cual nos permite expresar la aceleracion en el punto $C$ por medio de la siguiente expresion:
\begin{equation*}
	\vec a_C = \vec a_B + \vec a_{C/B}
\end{equation*}
En el cual:
\[
 a_c (↓) =(a_B)_t (⦭ \theta) + (a_B)_n (⦫ \theta) + (a_{C/B})_t (⦪ \beta) + (a_{C/B})_n (⦯ \beta)
\]

Con esta ecuacion y por medio de las proyecciones en el eje horizontal de cada vector encontramos la siguiente relacion que describe a $\alpha_3$ en funcion de $\alpha_1$ y $\omega_1$:
\begin{equation*}
	0 = -\alpha_1 R\sin\theta + \omega_1^2 R\cos\theta
	    + \alpha_3 K\cos\beta - \omega_3^2 K\sin\beta
\end{equation*}

Despejando $\alpha_3$:
\begin{equation}
	\boxed{
		\alpha_3 = \frac{\omega_1^2 R\cos\theta + \alpha_1 R\sin\theta - \omega_3^2 K\sin\beta}{K\cos\beta}
	}
\end{equation}
\section{Aceleraciones centroidales}
Para el análisis cinético se requieren las aceleraciones de los centroides. Se asume cada eslabón como cuerpo rígido homogéneo (centroide en su centro geométrico salvo correcciones de masa).
\begin{enumerate}
\item \textbf{Eslabón $O_1B$}

Centroide a distancia $\lambda$ de $O_1$. Aceleraciones tangencial y normal:
\begin{align}
\bar{a}_{O_{1x}} &= -\big(\alpha_1 \lambda \sin\theta + \omega_1^2 \lambda \cos\theta\big)\\
\bar{a}_{O_{1y}} &= \alpha_1 \lambda \cos\theta - \omega_1^2 \lambda \sin\theta
\end{align}

\item \textbf{Eslabón $O_2A$}

Centroide a $r/2$ de $O_2$:
\begin{align}
\bar{a}_{O_{2x}} &= -\big(\alpha_2 \tfrac{r}{2} \sin\phi + \omega_2^2 \tfrac{r}{2} \cos\phi\big)\\
\bar{a}_{O_{2y}} &= \alpha_2 \tfrac{r}{2} \cos\phi - \omega_2^2 \tfrac{r}{2} \sin\phi
\end{align}

\item \textbf{Martillo $C$}

Traslación pura vertical:
\begin{align}
\bar{a}_C &= \ddot{y}_c
\end{align}

\item \textbf{Eslabón $BC$}

Centroide a $K/2$ de $B$ con movimiento plano general:
\begin{align}
\bar{a}_{BC,x} &= -\frac{K}{2}\big(\alpha_3 \cos\beta + \omega_3^2 \sin\beta\big)\\
\bar{a}_{BC,y} &= -\ddot{y}_c + \frac{K}{2}\big(\alpha_3 \sin\beta - \omega_3^2 \cos\beta\big)
\end{align}
\end{enumerate}

\subsection{Momentos de inercia centroidales de los eslabones}

En el análisis cinético se requieren los momentos de inercia de los eslabones en el plano del mecanismo (eje $z$ saliendo del plano). Puesto que los eslabones se fabrican como prismas rectangulares (longitud $L$ y ancho en el plano $b$), el momento de inercia respecto al centroide $G$ alrededor del eje $z$ es:

\begin{equation}
\boxed{\; I_G = \frac{1}{12}\, m\, (L^2 + b^2) \;}
\end{equation}

Aplicando a cada eslabón del mecanismo:

\paragraph{Eslabón $O_1B$ (longitud $R$, ancho $b_R$, masa $m_R$).}
Momento de inercia centroidal y respecto al centro de area sin la ranura $O_1$:


\begin{align}
{I}_{o_1} &= \boxed{\tfrac{1}{12}\, m_R\, (R^2 + b_R^2 - R'^2 - b_R'^2) }
\end{align}

Con este valor es posible encontrar el momento de inercia centroidal de masa usando el teorema de ejes paralelos
\begin{align}
\bar{I}_{o_1} &= \boxed{ {I}_{o_1} - m(\delta)^2 }
\end{align}
\noindent Donde: $R$ es la longitud total de la barra $O_1B$, $\delta$ es la distancia del centro geometrico sin ranura al centro de masa, $b_R$ su ancho en el plano; $R'$ es la longitud de la ranura a lo largo de la barra y $b_R'$ el ancho de esa ranura; $m_R$ es la masa efectiva del eslabón (masa de la barra menos el material retirado por la ranura).


\noindent Donde: $R$ es la longitud total de la barra $O_1B$, $b_R$ su ancho en el plano; $R'$ es la longitud de la ranura a lo largo de la barra y $b_R'$ el ancho de esa ranura; $m_R$ es la masa efectiva del eslabón (masa de la barra menos el material retirado por la ranura).

\paragraph{Eslabón $O_2A$ (longitud $r$, ancho $b_r$, masa $m_r$).}
Momento de inercia centroidal y respecto al pivote $O_2$:

\begin{align}
\bar{I}_{r} &= \boxed{\tfrac{1}{12}\, m_r\, (r^2 + b_r^2) }
\end{align}

\paragraph{Eslabón $BC$ (longitud $K$, ancho $b_K$, masa $m_K$).}
Cuando el balance de momentos se toma alrededor del centroide del eslabón $BC$, como en la ecuación $Z$-$BC$ del informe, se requiere el momento de inercia centroidal:
\begin{align}
\bar{I}_{K} &= \boxed{\tfrac{1}{12}\, m_K\, (K^2 + b_K^2) }
\end{align}

% ============================================================
% SECCIÓN 5: ANÁLISIS CINÉTICO
% ============================================================
\section{Análisis cinético}
Con las variables cinemáticas se plantean ecuaciones de cuerpo rígido para cada eslabón (direcciones $X$, $Y$ y momentos alrededor de pivotes o centroides). Se emplean las fuerzas: reacciones $O_{1x}$, $O_{1y}$, $O_{2x}$, $O_{2y}$, fuerza de pasador $A$, reacciones en $B$ ($B_x,B_y$), fuerza vertical en martillo $C$ y par motor $\tau$.
\subsection{Ecuaciones de movimiento de newton -}
\subsection*{Ecuaciones de movimiento martillo C}
En el martillo C solo puede moverse verticalmente, ademas de que esta solo se mueve en traslacion y por tanto su aceleracion angular es 0.
\begin{enumerate}
    \item X - C:
    \begin{equation}
        C_x + N = 0 \quad \Rightarrow \quad
        C_x = -N
    \end{equation}
    \item Z - C:
    \begin{equation}
        -xC_x-yf_r = 0 \quad \Rightarrow \quad N(x-Y(\mu_K)) = 0
    \end{equation}
    esta relacion debido a que todos los demas valores no pueden ser 0 nos indica que $N = C_x = 0$
    \item Y - C:
    \begin{equation}
        W_c - C = m_c \bar{a}_c \quad \Rightarrow \quad C = m_c (g - \bar{a_c})
    \end{equation}
\end{enumerate}
\subsection*{Ecuaciones de movimiento del eslabon BC}
El eslabon BC presenta movimiento de plano general y por tanto tiene aceleracion angular diferente de 0 y aceleracion lineal tanto vertical como horizontal.
\begin{enumerate}
    \item X - BC:
    \begin{equation}
        B_x = m_{BC} \cdot (\bar{a}_{BC})_x
    \end{equation}

    \item Y - BC:
    \begin{equation}
        B_y - W_R - C = m_{BC} \cdot (\bar{a}_{BC})_y \quad \Rightarrow \quad B_y = m_{BC}(g + (\bar{a}_{BC})_y) + C
    \end{equation}

    \item Z - BC:
    \begin{equation}
        \frac{K}{2}(B_y \sin\beta - B_x \cos\beta + C\sin(\beta)) = \bar{I_{BC}} \cdot \alpha_3
    \end{equation}
\end{enumerate}
\subsection*{Ecuaciones de movimiento eslabon O1-B}
El eslabon O1-B no presenta restriccion en la direccion x e y, ademas que su aceleracion angular es diferente de 0. Es importante resaltar que el punto a debido a ser la reaccion de la ranura se sabe que es normal a la barra O1-B y $\lambda$ representa la distancia de o1 al centroide
\begin{enumerate}
    \item X - $O_1$:
    \begin{equation}
        O_{1x} - A\sin\theta - B_x = m_{o_1}(\bar{a}_{o_1})_x \quad \Rightarrow \quad O_{1x} - A\sin\theta = m_{o_1}(\bar{a}_{o_1})_x + B_x
    \end{equation}

    \item Y - $O_1$:
    \begin{equation}
        O_{1y} + A\cos\theta - B_y - W_{o_1} = m_{o_1}(\bar{a}_{o_1})_y \quad \Rightarrow \quad O_{1y} + A\cos\theta = m_{o_1}(g + \bar{a}_{o_1,y})+ B_y
    \end{equation}

    \item Z - $O_1$:
    \begin{equation}
        \left(L - \lambda\right)A + \lambda(-B_y \cos\theta  + B_x \sin\theta - O_{1y} \cos\theta + O_{1x} \sin\theta) = \bar{I_{o_1}} \cdot \alpha_1  
    \end{equation}
\end{enumerate}
\subsection*{Ecuaciones de movimiento eslabon O2-A}
La barra O2-A no presenta restriccion en la direccion x e y, ademas que su aceleracion angular es diferente de 0 y se le es impreso un par por parte del motor ($\tau$).
\begin{enumerate}
    \item X - $O_2$:
    \begin{equation}
        O_{2x} + A\sin\phi = m_r(\bar{a}_{o_2})_x
    \end{equation}

    \item Y - $O_2$:
    \begin{equation}
        O_{2y} - A\cos\phi - W_r= m_r(\bar{a}_{o_2})_y \quad \Rightarrow \quad O_{2y} - A\cos\phi = m_r((\bar{a}_{o_2})_y +g)
    \end{equation}

    \item Z - $O_2$:
    \begin{equation}
        \tau - \frac{r}{2}\cos({\phi - \theta}) + \frac{r}{2}(-O_{2y}\cos\phi + O_{2x}\sin\phi) = \bar{I_{o_2}} \cdot \alpha_2
    \end{equation}

\end{enumerate}

\subsection{Sistema reducido para resolución secuencial}
Dadas las 12 ecuaciones presenadas es posible reducir el sistema a 4 ecuaciones mediante remplazar $O_2x$ y $O_2y$ y resolver las ecuaciones que tienen solucion directa:
\begin{center}
\[
C = m_c \bigl(g - \bar{a}_c\bigr)
\]
\[
B_x = m_{BC}\,(\bar{a}_{BC})_x
\]
\[
B_y = m_{BC}\bigl(g + (\bar{a}_{BC})_y\bigr) + C
\]
\[
O_{2x} = m_r\,\bar{a}_{o_2x} - A\sin\theta
\]
\[
O_{2y} = m_r\bigl(\bar{a}_{o_2y} + g\bigr) + A\cos\theta
\]
\[
O_{1x} - A\sin\theta = m_{o1}\,\bar a_{O_{1x}} + B_x
\]
\[
O_{1y} + A\cos\theta = m_{o1}\bigl(\bar a_{O_{1y}} + g\bigr) + B_y
\]
\[
\lambda\sin\theta\,O_{1x} - \lambda\cos\theta\,O_{1y} + (L-\lambda)A
= \bar I_{o1}\,\alpha_1 - \lambda\bigl(B_x\sin\theta - B_y\cos\theta\bigr)
\]
\[
-2\cos(\phi-\theta)\,A + \frac{2}{r}\tau
= \frac{2}{r}\,\bar I_{o2}\,\alpha_2 + m_{o2}\bigl(g + \ddot{a}_{O_{2y}} - \ddot{a}_{O_{2x}}\bigr)
\]
\end{center}

\subsection{Potencia y eficiencia}
La potencia de entrada y salida se estiman como:
\begin{align*}
P_{in} &= \tau\,\omega_2, & P_{out} &= C\,\dot{y}_c, & \eta &= \frac{P_{out}}{P_{in}}\times 100\%.
\end{align*}

\section{Fase de Diseño Detallado}

La fase de diseño Detallado se divide en varias etapas, cabe realizar la aclaracion de que el diseño se baso en gran parte en la fudamentacion matematica: 

\subsection{Objetivo y Alcance}:

En esta estapa se define que se va a entregar(planos, ensamble, simulaciones dinámicas, etc), ademas de comprobar de manera practica y simulada los requisitos funcionales del prototipo. 

\subsection{Metodología y Entregables}

\subsubsection{bocetos}

La fase inicial de diseño se centra en la elaboración del diagrama conceptual del prototipo. En esta etapa se definen la geometría, las dimensiones y los conjuntos, así como la viabilidad funcional y constructiva de la solución propuesta. Este ejercicio preliminar, realizado antes del modelado detallado, permite estructurar un proceso de diseño secuencial y reproducible, identificar restricciones de fabricación y evaluar alternativas de ensamblaje. A continuación se adjuntan las imágenes de algunos bocetos

\begin{figure}[H]
  \centering
  \includegraphics[width=0.35\textwidth]{imagenes/Bocetos.png}
  \caption{Bocetos de Elementos pertenecientes al prototipado en su fase incial.}
  \label{fig:bocetos}
\end{figure}

\subsubsection{modelados}

Posteriormente a la fase de bocetos se realizó el modelado en 3D empleando Autodesk Inventor. Los bocetos iniciales fueron recreados como modelos sólidos, lo que permitió comprender en detalle la geometría de las piezas y evaluar su factibilidad de fabricación. El modelado facilitó además la creación y verificación del ensamblaje, la detección temprana de interferencias o errores de diseño y la preparación de los archivos para fabricación. Adicionalmente se efectuaron simulaciones de movimiento (simulación dinámica), que permiten estudiar de forma realista el comportamiento cinemático del mecanismo durante su operación y validar restricciones y recorridos antes de fabricar prototipos.

A continuación se adjuntan imágenes representativas de los modelos y del ensamblaje.

\begin{figure}[H]
  \centering
  \includegraphics[width=0.3\textwidth]{imagenes/Modelados.png}
  \caption{Modelados de piezas por medio de Software de diseño CAD Autodesk Inventor. }
  \label{fig:modelados}
\end{figure}

\begin{figure}[H]
  \centering
  \includegraphics[width=0.25\textwidth]{imagenes/Ensamble.png}
  \caption{Ensamble de componentes por medio de Software de diseño CAD Autodesk Inventor.}
  \label{fig:ensamble_cad}
\end{figure}

Para ver los todos los modelados y simulaciones vease la sección $ANEXOS$

\subsubsection{Planos}

Una vez completados los modelos sólidos de los elementos del prototipo, se procedió a la elaboración de los planos de fabricación. Los planos son documentos esenciales para la producción de los componentes; por ello se elaboraron siguiendo las normas técnicas de dibujo (NTC) aplicables en Colombia y se incorporaron las tolerancias pertinentes al proceso de manufactura previsto. Se verificó la concordancia y el sentido de las cotas, se aplicaron ajustes de tolerancia por proceso y se realizaron múltiples revisiones para garantizar el correcto dimensionamiento y montaje de las piezas.

A continuación se adjuntan algunos de los planos desarrollados:

\begin{figure}[H]
  \centering
  \includegraphics[width=0.4\textwidth]{imagenes/Planos.png}
  \caption{Planos del Mecanismo de retorno rapido tipo martillo.}
  \label{fig:planos}
\end{figure}

Cabe aclarar que en la imagen anterior no están todos los planos, si desea ver los planos en su totalidad diríjase a la sección de "ANEXOS"


\section{Análisis de Construcción y Pruebas del modelo}

Una vez concluida la fase de modelado, en la cual se verificó mediante simulación el correcto funcionamiento cinemático del mecanismo, se procedió a la fabricación del prototipo físico utilizando tecnología de manufactura aditiva. Como material base se empleó filamento de PLA de 1.75\,mm de diámetro, seleccionado por su facilidad de impresión, rigidez adecuada y bajo costo, lo que lo hace idóneo para una etapa de validación funcional.

\medskip

Tras la impresión de las piezas, se llevó a cabo el ensamble estructural del prototipo. La base principal fue unida a su complemento mediante resina epóxica, asegurando la rigidez necesaria para soportar las cargas del sistema. Durante el proceso de secado del adhesivo, se procedió al montaje de las articulaciones, utilizando ejes también fabricados mediante impresión 3D, los cuales fueron restringidos con tapas de cierre para evitar desplazamientos axiales y garantizar la correcta alineación durante la operación.

\medskip

Cabe mencionar que algunos de estos ejes requirieron lijado y ajuste dimensional, debido a pequeñas imperfecciones derivadas del proceso de impresión, con el fin de asegurar un movimiento fluido y sin fricción excesiva.

\medskip

Una vez completada la unión de la base y el soporte, se procedió a instalar el brazo articulado y el conjunto del mecanismo en posición vertical, configuración necesaria para reproducir el comportamiento de un martillo de forja y permitir el desplazamiento lineal descendente del percutor.

\medskip

Finalmente, se realizó la prueba funcional del sistema mediante la operación manual de la manivela. Durante la experimentación se observó que, dependiendo del sentido de giro, el mecanismo exhibía el comportamiento característico del sistema Whitworth:
\begin{itemize}
    \item Giro horario: fase de avance rápido y retorno lento.
    \item Giro antihorario: fase de avance lento y retorno rápido.
\end{itemize}

\medskip

Para comprobar su desempeño práctico, se colocó una muestra de plastilina bajo el martillo, observándose una deformación visible al accionar la manivela en sentido horario, confirmando así que el prototipo cumplía su propósito funcional y cinemático de transformar el movimiento rotacional manual en un impacto lineal controlado.

\begin{figure}[H]
  \centering
  \begin{subfigure}[b]{0.4\textwidth}
    \centering
    \includegraphics[width=\textwidth]{imagenes/Ensamble final.jpeg}
    \caption{Fase de ensamblaje del mecanismo}
    \label{fig:ensamble_final}
  \end{subfigure}
  \hfill
  \begin{subfigure}[b]{0.35\textwidth}
    \centering
    \includegraphics[width=\textwidth]{imagenes/Ensamble presentado.jpeg}
    \caption{Pruebas de funcionamiento}
    \label{fig:pruebas_funcionamiento}
  \end{subfigure}
  \caption{Fotografías del prototipo físico: ensamblaje y validación experimental}
  \label{fig:prototipo_fotos}
\end{figure}
\section{Análisis de los resultados teóricos}

Con los parámetros geométricos especificados se realizó un análisis numérico del mecanismo a lo largo de múltiples ciclos.

\subsection*{Posición vertical del martillo}

El sistema presenta un recorrido total de aproximadamente \textbf{206 mm} (desde $y_c \approx -162.4$ mm hasta $y_c \approx 43.3$ mm), adecuado para aplicaciones de martilleo ligero. Se observa la asimetría temporal característica: fase de descenso lenta (trabajo) y fase de ascenso rápida (retorno).

\begin{figure}[H]
    \centering
    \includegraphics[width=0.45\textwidth]{salidas/yc_vs_t.png}
    \caption{Posición vertical $y_c(t)$ del martillo: carrera de 206 mm con asimetría temporal.}
    \label{fig:yc_ciclo}
\end{figure}

\subsection*{Aceleración vertical}

La aceleración del martillo presenta valores extremos de $\ddot{y}_c \approx -55.9$ m/s² durante el descenso y $\ddot{y}_c \approx 2.3$ m/s² en el retorno, con un rango total de aproximadamente \textbf{58.2 m/s²}. El pico negativo significativo durante la fase de trabajo indica las cargas inerciales máximas que experimentan las uniones

\begin{figure}[H]
    \centering
    \includegraphics[width=0.5\textwidth]{salidas/ac_vs_t.png}
    \caption{Aceleración vertical $\ddot{y}_c$ del martillo.}
    \label{fig:ac_ciclo}
\end{figure}

\subsection*{Par motor requerido}
La variación del par $\tau$ respecto al tiempo nos permite visualizar la ida y vuelta en el martillo (esto gracias a la simetria que se puede notar en la grafica). Esto nos demuestra que para mantener un cierto valor deseado en la aceleracion centroidal del martillo c es necesario a lo largo del funcionamiento del mecanismo aplicar un par variable.
\begin{figure}[H]
    \centering
    \includegraphics[width=0.5\textwidth]{salidas/tau_vs_t.png}
    \caption{Par motor instantáneo $\tau$ necesario para mantener velocidad constante.}
    \label{fig:tau_ciclo}
\end{figure}

\section{Análisis sobre el funcionamiento del prototipo}

A partir de las simulaciones dinámicas realizadas en Autodesk Inventor y de los análisis numéricos desarrollados en Python, se verificó el correcto funcionamiento del prototipo en condiciones operativas previstas. Tras la impresión y el ensamblaje de las piezas, se llevaron a cabo pruebas experimentales que corroboraron las predicciones teóricas: los recorridos y restricciones del mecanismo se ajustaron al diseño, no se presentaron interferencias, y el comportamiento dinámico observado (trayectorias, rangos de giro y tiempos de respuesta) se mantuvo dentro de las tolerancias establecidas en el plan de diseño. Los resultados confirman la validez del modelo y permiten avanzar hacia la fase de optimización y ensayo a mayor escala.

\section{Conclusiones y recomendaciones}
\begin{enumerate}
  \item El mecanismo diseñado cumple su función de transformar el movimiento rotacional continuo en un recorrido alternado asimétrico; las mediciones del prototipo confirman las tendencias previstas por el modelo numérico.
  \item Los cálculos cinemáticos y dinámicos, junto con las verificaciones locales por elementos finitos, muestran valores reales; no obstante, es necesario revisar tolerancias y pares de apriete en las uniones para mitigar concentraciones de esfuerzo.
  \item En lugar del hueco interior para contrapeso, se recomienda emplear un sistema de fijación por succión (chupas) al suelo: esta solución facilita la estabilidad durante el impacto, permite ajustes de masa y simplifica mantenimiento y montaje.
  \item Mejoras geométricas puntuales —filetes en aristas, aumento localizado de sección en transiciones y control de radios— incrementan la vida a fatiga sin elevar significativamente el costo de fabricación.
  \item Es conveniente documentar y aplicar pares de apriete y secuencias de montaje para asegurar la reproducibilidad dinámica del prototipo.
  \item La instrumentación de ensayo debe contemplar un acelerómetro con rango apropiado a los picos medidos, muestreo $\geq 10\cdot f_{\max}$ y filtrado anti-aliasing para mejorar la correlación entre simulación y ensayo.
  \item En conjunto, el concepto es válido y reproducible; las medidas sugeridas (fijación mediante chupas, ajustes geométricos y control de montaje e instrumentación) aumentan la robustez y facilitan iteraciones futuras.
\end{enumerate}

% ===== REFERENCIAS =====



\addcontentsline{toc}{section}{Referencias}
\begin{thebibliography}{99}


\bibitem{beer}
Ferdinand P. Beer, E. Russell Johnston Jr., William E. Clausen, "Dinámica para ingeniería", McGraw-Hill, última edición disponible.

\bibitem{hal-01715664}
Autor desconocido, "Vector loop approach for mechanism analysis," HAL Open Science, 2018. Disponible en: https://hal.science/hal-01715664/document

\bibitem{stix2}
STIX Fonts Project, "STIX Two Math," 2020. Disponible en: https://stixfonts.org

\bibitem{mecapedia}
Mecapedia, "Mecanismo de retorno rápido de Whitworth," Universitat Jaume I. Disponible en: https://www.mecapedia.uji.es/pages/mecanismo\_de\_retorno\_rapido\_de\_Whitworth.html

\end{thebibliography}

\section*{Anexos}
\addcontentsline{toc}{section}{Anexos}

En el siguiente enlace se ha dispuesto una carpeta en Google Drive que reúne la totalidad de los anexos referenciados en este documento. La carpeta contiene planos de fabricación, modelos CAD, códigos fuente, gráficas, imágenes y demás archivos auxiliares, organizados por tipo y con documentación de soporte (README) para facilitar su consulta. Si requiere permisos de acceso, por favor solicítelos al autor del documento:

\begin{itemize}
    \item \textbf{Repositorio de archivos (Google Drive):} \\
    \url{https://drive.google.com/drive/folders/1992kkEoi9HfHHkotnGhMryOI3XGjRC80?usp=drive_link}
    
    \item \textbf{video presentacion proyecto:} \\
    \url{https://youtu.be/QGrWv6MqCTA}

\end{itemize}

\end{document}
