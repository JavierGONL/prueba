% !TeX program = lualatex
\documentclass{article}

\usepackage[spanish]{babel}
\usepackage{fontspec}
\usepackage{unicode-math}
\usepackage{graphicx}
\usepackage{float} % Para posicionamiento exacto de figuras con [H]
\usepackage{microtype} % Microtipografía: mejor espaciado y cortes
\usepackage{caption} % Control de espaciado en captions
\usepackage{geometry}
\geometry{a4paper, margin=2.5cm}
\usepackage{ragged2e} % Para el entorno justify
\usepackage{abstract} % Para el entorno abstract

\renewcommand{\contentsname}{Contenido}
\renewcommand{\abstractname}{Resumen}

% Ajustes globales de espaciado alrededor de figuras y captions
\setlength{\intextsep}{8pt plus 2pt minus 2pt}
\setlength{\textfloatsep}{8pt plus 2pt minus 2pt}
\setlength{\floatsep}{8pt plus 2pt minus 2pt}
\captionsetup{aboveskip=6pt,belowskip=6pt}

\setmainfont{Latin Modern Roman}

% Usa STIX Two Math que tiene los símbolos geométricos
\setmathfont{STIXTwoMath-Regular.otf}[
  Path = C:/Users/lunit/AppData/Local/Programs/MiKTeX/fonts/opentype/public/stix2-otf/,
  BoldFont = STIXTwoMath-Regular.otf
]

\usepackage{newunicodechar}

% ==== Definición de símbolos geométricos especiales ====
% Código Unicode U+29A8 a U+29AF
\newcommand{\angulese}{\ensuremath{\symbol{"29A8}}} % ⦨
\newunicodechar{⦨}{\angulese}

\newcommand{\anglees}{\ensuremath{\symbol{"29A9}}} % ⦩
\newunicodechar{⦩}{\anglees}

\newcommand{\anglenesw}{\ensuremath{\symbol{"29AA}}} % ⦪
\newunicodechar{⦪}{\anglenesw}

\newcommand{\anglensw}{\ensuremath{\symbol{"29AB}}} % ⦫
\newunicodechar{⦫}{\anglensw}

\newcommand{\measuredangleleft}{\ensuremath{\symbol{"29AC}}} % ⦬
\newunicodechar{⦬}{\measuredangleleft}

\newcommand{\measuredangleright}{\ensuremath{\symbol{"29AD}}} % ⦭
\newunicodechar{⦭}{\measuredangleright}

\newcommand{\sphericalangleup}{\ensuremath{\symbol{"29AE}}} % ⦮
\newunicodechar{⦮}{\sphericalangleup}

\newcommand{\sphericalangledown}{\ensuremath{\symbol{"29AF}}} % ⦯
\newunicodechar{⦯}{\sphericalangledown}
% =======================================================

\begin{document}

% ===== PORTADA =====
\begin{titlepage}
    \centering
    
    % Logo de la Universidad Nacional de Colombia 
    \includegraphics[width=4.5cm]{imagenes/escudo_unal.png}
    
    \vspace{1.5cm}
    
    {\LARGE\bfseries Diseño conceptual y análisis dinámico de un sistema de martillo accionado por un mecanismo de retorno rápido \par}
    
    \vspace{1.5cm}
    

    \vspace{2cm}
    
    {\large\bfseries Kevin Javier Gonzalez Luna\par}
    {\large  kegonzalezl@unal.edu.co\par}
    
    \vspace{1cm}
    
    {\large\bfseries Ivan Felipe Malucho Suarez \par}
    {\large imaluche @unal.edu.co\par}
    
    \vspace{1cm}
    
    {\large\bfseries Juan David Hernández Daza\par}
    {\large jhernandezda@unal.edu.co\par}

    \vspace{1cm}
    
    {\large\bfseries Joseph Nicolas Mahecha Cruz\par}
    {\large jomahecha@unal.edu.co\par}
    \vfill
    
    {\large Nelson Arzola de la Pe{\~n}a\\
    \vspace{1cm}
    Universidad Nacional de Colombia,\\
    Bogotá D.C.\\
    Facultad de Ingeniería.\\
    2025.\par}
    
\end{titlepage}

% ===== INICIO DEL CONTENIDO =====
\newpage

% ===== RESUMEN =====
\addcontentsline{toc}{section}{Resumen}

\begin{abstract}
El presente proyecto desarrolla el diseño y modelado cinemático de un mecanismo de retorno rápido tipo Whitworth, modificado para operar como un martillo de impacto vertical. El objetivo principal consiste en transformar un movimiento rotacional uniforme, proporcionado por un motor eléctrico con reductor, en un movimiento alternativo de avance y retroceso con diferentes velocidades angulares efectivas, permitiendo un descenso lento y un retorno rápido del elemento percutor.

El rediseño propuesto introduce una variación estructural respecto al mecanismo clásico: el rodillo deslizante tradicional es reemplazado por un tornillo pasante que transmite directamente la fuerza entre la manivela motriz y la palanca ranurada. Esta modificación reduce el número de componentes, simplifica la fabricación y mantiene la eficiencia cinemática del sistema. La energía rotacional del motor se transmite desde el eje de entrada $O_2$ a la palanca oscilante pivotada en $O_1$, la cual a su vez impulsa una biela articulada que acciona un bloque deslizante. Dicho bloque, fabricado en acero A36, cumple la función de martillo y se desplaza guiado por dos varillas lineales, ejecutando un golpe controlado durante la fase de avance.

El mecanismo completo se inscribe en un volumen máximo de $60 \times 60 \times 15 \ \text{cm}$ y fue analizado mediante el método del lazo vectorial, considerando relaciones angulares, velocidades y aceleraciones instantáneas. Los resultados obtenidos muestran una correspondencia directa entre la velocidad angular de entrada y la aceleración lineal del martillo, validando el comportamiento asimétrico característico del retorno rápido. Este diseño demuestra una alternativa funcional, compacta y didáctica para el estudio de la conversión de movimiento rotacional en lineal alternativo, aplicable tanto en entornos académicos como en desarrollos de prototipos mecánicos ligeros.
\end{abstract}

\noindent\textbf{Palabras clave:} mecanismo de retorno rápido de Whitworth; martillo de impacto vertical; palanca ranurada; pasador deslizante; lazo vectorial; cinemática de mecanismos.

\vspace{0.5cm}

\newpage
% ===== TABLA DE CONTENIDO =====
\tableofcontents
\newpage


% ===== INTRODUCCIÓN =====
\section{Introducción}

\begin{justify}
    (falta arreglarlo)
\end{justify}

% ===== FUNDAMENTACIÓN TEÓRICA =====
\section{Fundamentación teórica}

El análisis de mecanismos planos requiere el uso de conceptos fundamentales de cinemática y cinética, entre estos para el análisis del mecanismo se harán uso de:

\subsection{Movimiento de plano general}

El movimiento de plano general describe los valores de velocidad y aceleración de un punto de un sólido rígido por medio de las siguientes ecuaciones: 

\begin{equation}
\vec{V}_b = \vec{V}_a + \vec{V}_{b/a} = \vec{V}_a + (\vec{\omega} \times \vec{r}_{b/a})
\end{equation}

\begin{equation}
\vec{a}_b = \vec{a}_a + \vec{a}_{b/a} = \vec{a}_a + (\vec{\alpha} \times \vec{r}_{b/a}) + (\vec{\omega} \times (\vec{\omega} \times \vec{r}_{b/a}))
\end{equation}

Donde: 
\begin{itemize}
    \item $\vec{\omega}$ y $\vec{\alpha}$ representan la velocidad y aceleración angular respectivamente, son iguales a la primera y segunda derivada del angulo de rotacion respecto al tiempo ($\dot{\theta}, \ddot{\theta}$)
    \item $\vec{r}_{b/a}$ representa el vector que une $\vec{r_a}$ con $\vec{r_b}$
\end{itemize}
\subsection{Movimiento plano de una particula relativa a un sistema de referencia en rotacion}
Cuando se presentan casos donde el objeto se mueve dentro de un sistema en rotacion (como es en nuestro caso el pasador que se mueve linealmente sobre la barra $O1B$) es necesario usar las siguiente ecuaciones para encontrar la variables cinematicas de un punto p dado: 

\begin{equation}
\vec{V}_p = \vec{V}'_p + \vec{V}_{p/f} 
\end{equation}

\begin{equation}
\vec{a}_p = \vec{a}'_p + \vec{a}_{p/f} + a_{cor} 
\end{equation}
Donde: 
\begin{itemize}
    \item $\vec{V'_p}$ y $\vec{a'_p}$ representan las variables cinematicas que posee el cuerpo producto de la rotacion del sistema de referencia, se puede entender como un punto del sistema de referencia que coincide con p en todo momento
    \item $\vec{V}_{p/f}$ y $\vec{a}_{p/f}$ representan las variables cinematicas del cuerpo relativas al sistema, en nuestro caso al ser un pasador sigue la geometria de la barra ${r_b}$
    \item $\vec{a}_{cor}$ representa la aceleracion de coreolis, la cual debe ser igual a $2\vec{\Omega} \times \vec{V_{p/f}}$ y es propia del sistema en rotacion
\end{itemize}

\subsection{Método del lazo vectorial}

Este método analiza mecanismos a partir de ecuaciones vectoriales que cierran un circuito. Cada eslabón se modela como un vector de posición en un sistema de coordenadas global $XY$. Con las longitudes conocidas y los ángulos de entrada, se plantea el lazo de cierre para obtener relaciones geométricas entre las posiciones de los eslabones. Luego, dichas relaciones se derivan una y dos veces respecto al tiempo para obtener, respectivamente, las velocidades y las aceleraciones instantáneas[ref:https://hal.science/hal-01715664/document ].

En la práctica, el procedimiento consiste en:

\begin{itemize}
    \item Plantear el cierre vectorial y su proyección en los ejes $x$ y $y$.
    \item Resolver la geometría para las incógnitas instantáneas (p. ej., $\theta$, $\beta$, $y_c$).
    \item Derivar una vez para obtener las ecuaciones de velocidad y una segunda vez para las de aceleración.
\end{itemize}

\subsection{Ecuaciones de movimiento de un cuerpo rigido}

las ecuaciones de movimiento de un cuerpo rigido me relacionan los vectores de fuerza efectiva con las fuerzas de reaccion del mecanismo:

\begin{itemize}
    \item  $\sum \vec{F} = m \bar{a}$
    \item  $\sum \vec{M_G} = \dot{H_G}$
\end{itemize}

Donde:
\begin{itemize}
    \item $\sum \vec{F}$ es la suma de todas las fuerzas externas aplicadas al cuerpo.
    \item $\sum \vec{M_G}$ es la suma de los momentos respecto al centro de masa $G$.
    \item $\dot{H_G}$ es la derivada temporal del momento angular respecto al centro de masa. 
    \item $\dot{H_G} = \bar{I} \alpha$
\end{itemize}
Estas ecuaciones permiten analizar el movimiento de cuerpos rígidos sometidos a fuerzas y momentos, relacionando las fuerzas efectivas con las reacciones internas del mecanismo. Son la base para el estudio dinámico de sistemas mecánicos como el mecanismo de martillo presentado en este informe.

% ===== FASE DE DISEÑO CONCEPTUAL =====
\section{Fase de Diseño Conceptual}

El mecanismo consiste en un sistema de barras articuladas accionado por un motor que produce un movimiento de martilleo. Los componentes principales son:

\begin{itemize}
    \item \textbf{Motor:} Proporciona rotación constante con velocidad angular $\omega_2$ y aceleración angular $\alpha_2 = 0$
    \item \textbf{Barra motriz ($O_2A$):} De longitud $r$, conectada al motor en $O_2$ y al punto $A$
    \item \textbf{Barra acoplada ($O_1B$):} De longitud $L$ variable aparente, articulada en $O_1$ y conectada en punto coincidente con $A$
    \item \textbf{Sistema de martillo:} Mecanismo articulado con barra de longitud constante $K$ que produce el movimiento de impacto
\end{itemize}

\subsection{Variables del sistema}

\begin{itemize}
    \item $\delta(t) = \omega_2 t$: Ángulo de la barra motriz (entrada del sistema)
    \item $\alpha(t)$: Ángulo de la barra acoplada (función de $\delta$)
    \item $L(\delta)$: Distancia variable entre articulaciones
    \item $\omega_1 = \dot{\alpha}$: Velocidad angular de la barra acoplada
    \item $X_E(t)$: Posición del martillo
    \item $R$, $D$, $K$: Parámetros geométricos del mecanismo
\end{itemize}

\subsection{Modelo de prototipo inicial}

\begin{justify}

El proyecto se centra en el diseño y en la construcción de un mecanismo de retorno rápido basado en el mecanismo de Whitworth, modificado específicamente para funcionar como un martillo de forja. El objetivo principal es convertir el movimiento rotacional uniforme del motor eléctrico en un movimiento lineal alternativo con cinemática asimétrica, característica fundamental para que la fase de avance lento y controlado sea efectiva a la hora de transferir energía de impacto de forma eficiente. En la fase de retorno rápido, se busca reducir el tiempo muerto del ciclo, incrementando así la productividad del sistema.

\begin{figure}[H]
\centering
\includegraphics[width=0.4\textwidth]{{imagenes/diagrama algo mecanismo}.png}
\caption{Modelo esquemático del mecanismo de retorno rápido tipo Whitworth modificado.}
\label{fig:mecanismo_prototipo}
\end{figure}

\vspace{0.5em}

La innovación central radica en una reinterpretación espacial y constructiva del mecanismo clásico. A diferencia de la configuración tradicional, el sistema ha sido reorientado noventa grados, posicionando el plano de movimiento en un eje vertical. Esta modificación permite que el bloque deslizante final actúe directamente como un martillo, aprovechando su carrera lineal para el golpeo. El principio de funcionamiento se basa en la transmisión de movimiento desde una manivela motriz, solidaria al eje del motor en el punto $O_2$, hacia una palanca oscilante ranurada que pivota en $O_1$.

\vspace{0.5em}

Una solución de diseño clave sustituye el rodillo convencional por un tornillo pasante que actúa como un elemento de unión y guía. Este tornillo se desliza dentro de la ranura longitudinal de la palanca, estableciendo un contacto lineal directo que transmite el esfuerzo. Esta simplificación no solo facilita la fabricación y reduce el número de componentes, al eliminar el rodamiento intermedio, sino que también mantiene la esencia del mecanismo de deslizamiento que genera el efecto de retorno rápido.

\vspace{0.5em}

Finalmente, el movimiento angular oscilante de la palanca ranurada se transmite a través de una biela articulada al bloque deslizante o martillo. Este conjunto completa la transformación cinemática, guiando el martillo en su trayectoria vertical. Como resultado, se obtiene un dispositivo robusto y funcional que sincroniza perfectamente la entrada de rotación continua con la salida de impacto repetitivo.

\vspace{0.5em}

Cabe destacar que el modelado presentado se encuentra sujeto a posibles modificaciones, ya que aún está pendiente el diseño definitivo de la estructura o soporte del mecanismo. Dicho componente será determinante para definir la disposición final de los ejes, la ubicación del motor y los puntos de anclaje, por lo que es posible que se requieran ajustes dimensionales o constructivos en el modelo actual una vez se integre el sistema de soporte completo.

\end{justify}

% ===== MODELOS TEÓRICOS =====
\section{modelo cinemático}

\subsection*{Parámetros geométricos y operativos}

\begin{figure}[H]
    \centering
    \includegraphics[width=0.6\textwidth]{{imagenes/diagrama mecanismo}.png}
    \caption{Diagrama geométrico del mecanismo con referencias de dimensiones (r, d, R, K, D).}
    \label{fig:diagrama_mecanismo}
\end{figure}

\vspace{0.5em}
 
\addcontentsline{toc}{subsection}{Parámetros geométricos y operativos (SI)}
Los valores usados en el análisis corresponden al diseño actual del prototipo y se expresan en unidades SI:
\begin{align*}
 r &= 0.070\;\text{m} & d &= 0.110\;\text{m} & R &= 0.200\;\text{m} & \delta = 0.00665\\
 K &= 0.070\;\text{m} & D &= 0.17 \;\text{m} & \lambda &= 0.1093458\;\text{m} & g &= 9.81\;\text{m/s}^2 \\
 \omega_2 &= 4.00\;\text{rad/s} & \alpha_2 &= 0.00\;\text{rad/s}^2
\end{align*}
Se asume motor de velocidad angular constante ($\alpha_2=0$) y todos los ángulos en radianes. El valor de $\lambda$ corresponde a la distancia del pivote $O_1$ al centroide del eslabón $O_1B$.

\subsection{Relación geométrica entre $\theta$ y $\phi$}

Inicialmente para el analisis cinemático es necesario encontrar expresiones que permitan describrir las distancias
e inclinaciones de todas las partes del mecanismo que conforman el sistema en funcion de una unica variable, para este fin y por medio de un analisis geometrico es posible establecer la siguiente relacion fundamental entre los ángulos por medio del triángulo formado por las longitudes $r$, $d$ y $L$. De esta manera obtenemos:

\begin{equation*}
L^2 = r^2 + d^2 - 2rd \cos(180^\circ - \phi)
\end{equation*}

Por medio de la propiedad trigonometrica  $\cos(180^\circ - \Phi) = -\cos(\phi)$, reducimos la expresion a:

\begin{equation*}
L = \sqrt{r^2 + d^2 + 2rd \cos(\phi)}
\end{equation*}

Donde L representara la distancia desde el punto O hasta el punto A. Sabiendo esto y usando la ley de senos es posible obtener la siguiente expresion:

\begin{equation*}
\frac{\sin(180^\circ - \Phi)}{L} = \frac{\sin(\Theta)}{r}
\end{equation*}

Usando la propiedad trigonometrica $\sin(180-\phi) = sin(\phi) $ y despejando r del lado derecho de la ecuacion obtenemos:

\begin{equation*}
\sin(\theta) = \frac{\sin(\phi) r}{L}
\end{equation*}


Sustituyendo $L$ en la expresion obtenemos:

\begin{equation}
\boxed{
\sin(\Theta) =
\frac{r \sin(\Phi)}
{\sqrt{r^2 + d^2 + 2rd \cos(\Phi)}}
}
\end{equation}

De esta forma se establece una Relacion directa entre el angulo $\theta$ y $\phi$.

\subsection{Velocidades angulares y relativas}

Con la geometria del sistema definimos la velocidad en el punto $a$ por medio de las restricciones cinematicas de cada una de las partes a las que pertenece.
En primera instancia sabemos que el punto $a$ respecto la barra $O_2A$ tiene velocidad y aceleración angulares, mas no presenta traslacion. Por tanto se puede plantear la siguiente ecuacion de rotacion respecto a un eje fijo:
\[
  \vec V_a = \boldsymbol{\omega}_2 \times \vec r,\; ⦭\, \Phi^\circ
\]
De igual manera planteamos al punto $a$ respecto a la barra $O_1B$, en la cual al generar un movimiento relativo a un sistema en rotacion es posible obtener la siguiente ecuacion vectorial para $ \vec V_a $:
\[
  \vec V_a = \vec V'_a + \vec V_{a/f}
\]


donde $\vec V'_a$ representa la velocidad producto de la rotacion del sistema mientras que $\vec V_{a/f}$ representa la velocidad relativa de $a$ en el sistema.

\begin{figure}[H]
    \centering
    \includegraphics[width=0.4\textwidth]{{imagenes/Poligono velocidades A}.png}
    \caption{Polígono de velocidades en el punto A.}
    \label{fig:poligono_velocidades_A}
\end{figure}

Analizando el diagrama y por medio del teorema de $\sin(\phi - \theta) = \frac{V_{a/f}}{V_a}$ obtenemos:

\begin{equation*}
    V_{a/f} = ( \omega_2\cdot r \cdot \sin(\phi - \theta))
\end{equation*}

Usando $\cos(\phi - \theta) = \frac{V_a'}{V_a}$  tambien obtenemos la siguiente expresion para la velocidad $V'_a$: 
\begin{equation*}
    V_a' = \cos(\Phi -\Theta) \, V_a
\end{equation*}

Finalmente a partir de esta ultima relacion, remplazando con sus valores producto de su rotacion respecto a un eje fijo y despejando para $\omega_1$, obtenemos:

\begin{equation}
    \boxed{
    \omega_1 = \frac{\omega_2 r}{L} \, \cos(\Phi -\Theta)
    }
\end{equation}

% ============================================================
% SECCIÓN 2: ANÁLISIS DE ACELERACIONES
% ============================================================

\subsection{Aceleraciones angulares y relativas}

De manera similar a en el analisis de velocidades tomamos a $a$ respecto a la barra $O_2A$ tiene velocidad y aceleración angular y no presenta traslación, debido a esto aplicamos nuevamente rotación respecto a eje fijo y obtenemos la siguiente expresion vectorial:
\[
  \vec a_a = (\alpha_2 \times \vec r) ,\; ⦭\, \Phi^\circ  + (\boldsymbol{\omega}_2 \times(\boldsymbol{\omega}_2 \times \vec r)) ,\; ⦫\, \Phi^\circ
\] 
Donde el primer termino representa la aceleracion tangencial de $a$ mientras el segundo representa su aceleracion normal.
Hecho esto ahora consideramos a como parte de la barra $O_1B$, obteniendo la sigueinte relacion:
\[
  \vec a_a = \vec a'_a + \vec a_{a/f} + \vec a_{cor}
\] 
Donde:

\begin{itemize}
    \item \[
  \vec a'_a =  (\alpha_1 \times \vec L) ,\; ⦫\, \theta^\circ  + (\boldsymbol{\omega}_1 \times(\boldsymbol{\omega}_1 \times \vec L)) ,\; ⦭\, \theta^\circ 
\] 
    \item \[
  \vec a_{cor} =  (2\boldsymbol{\omega}_1 \times \vec V_{a/f}) ,\; ⦮\, \theta^\circ \] 
    \item \[a_{a/f} =  \; ⦫\,\theta^\circ \]
\end{itemize}

Igualando ambas ecuaciones vectoriales para encontrar las incognitas requeridas proyectamos todos los vectores respecto a sus componentes rectangulares, obteniendo las siguientes ecuaciones para X e Y:
\begin{align*}
(a'_{a})_{t} \sin\theta + (a_{a/f}) \cos\theta 
&= \omega_2^2 r \cos(\Phi) + \alpha_2 r \sin(\Phi) 
+ 2\omega_1 V_{a/f} \sin\theta - \omega_1^2 L \cos\theta \\[8pt]
(a'_{a})_{t} \cos\theta - (a_{a/f}) \sin\theta 
&= \alpha_2 r \cos(\Phi) - \omega_2^2 r \sin(\Phi)
+ 2\omega_1 V_{a/f} \cos\theta + \omega_1^2 L \sin\theta
\end{align*}
Solucionando el sistema para $\alpha_1$ y $a_{a/f}$:

\begin{equation*}
\alpha_1 = \frac{\omega_2^2 r\sin(\theta-\Phi)+\alpha_2 r\cos(\theta-\Phi)+2\omega_1V_{a/f}}{L}
\end{equation*}

\begin{equation*}
a_{a/f} = \omega_2^2 r\cos(\Phi-\theta) + {\alpha}_2 r \sin(\Phi-\theta) - \omega_1^2 L
\end{equation*}

% ============================================================
% SECCIÓN 3: LAZO VECTORIAL PARA OBTENER VEL Y ACEL DEL MARTILLO
% ============================================================
\subsection{Lazo vectorial del martillo: posición, velocidad y aceleración}

Planteando el lazo vectorial para el mecanismo, se tiene:

% Lazo vectorial (ya existe la imagen en la carpeta 'imagenes')
\begin{figure}[H]
    \centering
    \includegraphics[width=0.4\textwidth]{{imagenes/Lazo vectorial}.png}
    \caption{Lazo vectorial del mecanismo que relaciona $O_1B$, el eslabón $BC$ y el desplazamiento vertical del martillo.}
    \label{fig:lazo_vectorial}
\end{figure}

Al proyectar en los ejes $x$ y $y$:

\begin{align*}
    X: D - R\cos\theta + K\sin\beta &= 0 \\
    Y: y_c + K\cos\beta - R\sin\theta &= 0
\end{align*}

Elevando ambas ecuaciones al cuadrado y sumando:

\begin{equation*}
    K^2 = {(R\cos\theta - D)}^{2} + {(R\sin\theta - y_c)}^{2}
\end{equation*}

\subsubsection{Resultado para la velocidad}

Dividiendo la expresion anterior, obtenemos:

\begin{align}
\boxed{\dot{y}_c = \frac{R\omega_1(D\sin(\theta) - y_c\cos(\theta))}{R\sin(\theta) - y_c}}
\end{align}

Esta expresión es válida siempre que $R\sin(\theta) \neq y_c$ (es decir, que el denominador no sea cero).

\subsubsection{Aceleración mediante lazo vectorial}

Derivando nuevamente la ecuación de velocidad para obtener la aceleración:

\begin{equation}
\ddot{y}_c = 
R \; 
\frac{
\big[ \dot{\omega}_1 A - \omega_1^2 B - \omega_1 \dot{y}_c \cos\theta \big] D_{en}
- \omega_1 A (\dot{y}_c + R \omega_1 \cos\theta)
}{
D_{en}^2
}
\end{equation}

\noindent donde:
\begin{align*}
A &= D\sin\theta - y_c\cos\theta,\\
B &= D\cos\theta + y_c\sin\theta,\\
D_{en} &= R\sin\theta - y_c.
\end{align*}

Nota: $\dot{\omega}_1 = \ddot{\theta} = \alpha_1$ es la aceleración angular.

% ============================================================
% SECCIÓN 4: CÁLCULO DE ω₃ Y α₃
% ============================================================
\section{Cinemática del eslabón $BC$}
\subsection{Ángulo $\beta$ del eslabón $K$}

Antes de obtener las variables cineticas del mecanismo es necesario preparar el terreno por medio de algunas relaciones cinematicas adicionales las cuales no fueron necesarias con anterioridad para encontrar los puntos de inetres.

\subsubsection{Inclinacion eslabón K}
Para encontrar el valor del angulo $\beta$ el cual describe la inclinacion del eslabón respecto a la vertical se hace uso de la proyeccion en X desarrollada por medio del lazo vectorial, la cual al despejar $\beta$ nos permite obtener la siguiente expresion:
\begin{equation*}
	\sin\beta = \frac{R\cos\theta - D}{K}, \qquad
	\beta = \arcsin\left(\frac{R\cos\theta - D}{K}\right)
\end{equation*}

\subsubsection{Ecuacion vectorial de velocidad en el punto $B$}

Usando movimiento del plano general podemos expresar la velocidad en el punto $C$ por medio de la siguiente expresion:
\begin{equation*}
	\vec v_C = \vec v_B + \vec v_{C/B}
\end{equation*}
En la cual:

\[
 V_c (↓) = \omega_1 \cdot R (⦭ \theta) + \omega_3 \cdot K (⦪\beta)
\]

Tomando en cuenta las inclinaciones proyectamos los vectores en el eje horizontal, obtienendo asi la siguiente relacion para $\omega_3$ y $\omega_1$:
\begin{equation*}
	0 = -\omega_1\,R\,\sin\theta - \omega_3\,K\,\cos\beta
\end{equation*}

Despejando $\omega_3$ de la ecuacion anterior obtenemos:
\begin{equation}
	\boxed{
		\omega_3 = -\,\frac{\omega_1\,R\,\sin\theta}{K\,\cos\beta}
	}
\end{equation}


\subsubsection{Ecuacion vectorial de aceleración en el punto $B$}

De manera similar a en el caso de las velocidad se tiene un movimiento de plano general, el cual nos permite expresar la aceleracion en el punto $C$ por medio de la siguiente expresion:
\begin{equation*}
	\vec a_C = \vec a_B + \vec a_{C/B}
\end{equation*}
En el cual:
\[
 a_c (↓) =(a_B)_t (⦭ \theta) + (a_B)_n (⦫ \theta) + (a_{C/B})_t (⦪ \beta) + (a_{C/B})_n (⦯ \beta)
\]

Con esta ecuacion y por medio de las proyecciones en el eje horizontal de cada vector encontramos la siguiente relacion que describe a $\alpha_3$ en funcion de $\alpha_1$ y $\omega_1$:
\begin{equation*}
	0 = -\alpha_1 R\sin\theta + \omega_1^2 R\cos\theta
	    + \alpha_3 K\cos\beta - \omega_3^2 K\sin\beta
\end{equation*}

Despejando $\alpha_3$:
\begin{equation}
	\boxed{
		\alpha_3 = \frac{\omega_1^2 R\cos\theta + \alpha_1 R\sin\theta - \omega_3^2 K\sin\beta}{K\cos\beta}
	}
\end{equation}
\section{Aceleraciones centroidales}
Para el análisis cinético se requieren las aceleraciones de los centroides. Se asume cada eslabón como cuerpo rígido homogéneo (centroide en su centro geométrico salvo correcciones de masa).
\begin{enumerate}
\item \textbf{Eslabón $O_1B$}

Centroide a distancia $\lambda$ de $O_1$. Aceleraciones tangencial y normal:
\begin{align}
\bar{a}_{O_{1x}} &= -\big(\alpha_1 \lambda \sin\theta + \omega_1^2 \lambda \cos\theta\big)\\
\bar{a}_{O_{1y}} &= \alpha_1 \lambda \cos\theta - \omega_1^2 \lambda \sin\theta
\end{align}

\item \textbf{Eslabón $O_2A$}

Centroide a $r/2$ de $O_2$:
\begin{align}
\bar{a}_{O_{2x}} &= -\big(\alpha_2 \tfrac{r}{2} \sin\phi + \omega_2^2 \tfrac{r}{2} \cos\phi\big)\\
\bar{a}_{O_{2y}} &= \alpha_2 \tfrac{r}{2} \cos\phi - \omega_2^2 \tfrac{r}{2} \sin\phi
\end{align}

\item \textbf{Martillo $C$}

Traslación pura vertical:
\begin{align}
\bar{a}_C &= \ddot{y}_c
\end{align}

\item \textbf{Eslabón $BC$}

Centroide a $K/2$ de $B$ con movimiento plano general:
\begin{align}
\bar{a}_{BC,x} &= -\frac{K}{2}\big(\alpha_3 \cos\beta + \omega_3^2 \sin\beta\big)\\
\bar{a}_{BC,y} &= -\ddot{y}_c + \frac{K}{2}\big(\alpha_3 \sin\beta - \omega_3^2 \cos\beta\big)
\end{align}
\end{enumerate}

\subsection{Momentos de inercia centroidales de los eslabones}

En el análisis cinético se requieren los momentos de inercia de los eslabones en el plano del mecanismo (eje $z$ saliendo del plano). Puesto que los eslabones se fabrican como prismas rectangulares (longitud $L$ y ancho en el plano $b$), el momento de inercia respecto al centroide $G$ alrededor del eje $z$ es:

\begin{equation}
\boxed{\; I_G = \frac{1}{12}\, m\, (L^2 + b^2) \;}
\end{equation}

Aplicando a cada eslabón del mecanismo:

\paragraph{Eslabón $O_1B$ (longitud $R$, ancho $b_R$, masa $m_R$).}
Momento de inercia centroidal y respecto al centro de area sin la ranura $O_1$:


\begin{align}
{I}_{o_1} &= \boxed{\tfrac{1}{12}\, m_R\, (R^2 + b_R^2 - R'^2 - b_R'^2) }
\end{align}

Con este valor es posible encontrar el momento de inercia centroidal de masa usando el teorema de ejes paralelos
\begin{align}
\bar{I}_{o_1} &= \boxed{ {I}_{o_1} - m(\delta)^2 }
\end{align}
\noindent Donde: $R$ es la longitud total de la barra $O_1B$, $\delta$ es la distancia del centro geometrico sin ranura al centro de masa, $b_R$ su ancho en el plano; $R'$ es la longitud de la ranura a lo largo de la barra y $b_R'$ el ancho de esa ranura; $m_R$ es la masa efectiva del eslabón (masa de la barra menos el material retirado por la ranura).


\noindent Donde: $R$ es la longitud total de la barra $O_1B$, $b_R$ su ancho en el plano; $R'$ es la longitud de la ranura a lo largo de la barra y $b_R'$ el ancho de esa ranura; $m_R$ es la masa efectiva del eslabón (masa de la barra menos el material retirado por la ranura).

\paragraph{Eslabón $O_2A$ (longitud $r$, ancho $b_r$, masa $m_r$).}
Momento de inercia centroidal y respecto al pivote $O_2$:

\begin{align}
\bar{I}_{r} &= \boxed{\tfrac{1}{12}\, m_r\, (r^2 + b_r^2) }
\end{align}

\paragraph{Eslabón $BC$ (longitud $K$, ancho $b_K$, masa $m_K$).}
Cuando el balance de momentos se toma alrededor del centroide del eslabón $BC$, como en la ecuación $Z$-$BC$ del informe, se requiere el momento de inercia centroidal:
\begin{align}
\bar{I}_{K} &= \boxed{\tfrac{1}{12}\, m_K\, (K^2 + b_K^2) }
\end{align}

% ============================================================
% SECCIÓN 5: ANÁLISIS CINÉTICO
% ============================================================
\section{Análisis cinético}
Con las variables cinemáticas se plantean ecuaciones de cuerpo rígido para cada eslabón (direcciones $X$, $Y$ y momentos alrededor de pivotes o centroides). Se emplean las fuerzas: reacciones $O_{1x}$, $O_{1y}$, $O_{2x}$, $O_{2y}$, fuerza de pasador $A$, reacciones en $B$ ($B_x,B_y$), fuerza vertical en martillo $C$ y par motor $\tau$.
\subsection{Ecuaciones de movimiento de newton -}
\subsection*{Ecuaciones de movimiento martillo C}
En el martillo C solo puede moverse verticalmente, ademas de que esta solo se mueve en traslacion y por tanto su aceleracion angular es 0.
\begin{enumerate}
    \item X - C:
    \begin{equation}
        C_x + N = 0 \quad \Rightarrow \quad
        C_x = -N
    \end{equation}
    \item Z - C:
    \begin{equation}
        -xC_x-yf_r = 0 \quad \Rightarrow \quad N(x-Y(\mu_K)) = 0
    \end{equation}
    esta relacion debido a que todos los demas valores no pueden ser 0 nos indica que $N = C_x = 0$
    \item Y - C:
    \begin{equation}
        W_c - C = m_c \bar{a}_c \quad \Rightarrow \quad C = m_c (g - \bar{a_c})
    \end{equation}
\end{enumerate}
\subsection*{Ecuaciones de movimiento del eslabon BC}
El eslabon BC presenta movimiento de plano general y por tanto tiene aceleracion angular diferente de 0 y aceleracion lineal tanto vertical como horizontal.
\begin{enumerate}
    \item X - BC:
    \begin{equation}
        B_x = m_{BC} \cdot (\bar{a}_{BC})_x
    \end{equation}

    \item Y - BC:
    \begin{equation}
        B_y - W_R - C = m_{BC} \cdot (\bar{a}_{BC})_y \quad \Rightarrow \quad B_y = m_{BC}(g + (\bar{a}_{BC})_y) + C
    \end{equation}

    \item Z - BC:
    \begin{equation}
        \frac{K}{2}(B_y \sin\beta - B_x \cos\beta + C\sin(\beta)) = \bar{I_{BC}} \cdot \alpha_3
    \end{equation}
\end{enumerate}
\subsection*{Ecuaciones de movimiento eslabon O1-B}
El eslabon O1-B no presenta restriccion en la direccion x e y, ademas que su aceleracion angular es diferente de 0. Es importante resaltar que el punto a debido a ser la reaccion de la ranura se sabe que es normal a la barra O1-B y $\lambda$ representa la distancia de o1 al centroide
\begin{enumerate}
    \item X - $O_1$:
    \begin{equation}
        O_{1x} - A\sin\theta - B_x = m_{o_1}(\bar{a}_{o_1})_x \quad \Rightarrow \quad O_{1x} - A\sin\theta = m_{o_1}(\bar{a}_{o_1})_x + B_x
    \end{equation}

    \item Y - $O_1$:
    \begin{equation}
        O_{1y} + A\cos\theta - B_y - W_{o_1} = m_{o_1}(\bar{a}_{o_1})_y \quad \Rightarrow \quad O_{1y} + A\cos\theta = m_{o_1}(g + \bar{a}_{o_1}_{y})+ B_y
    \end{equation}

    \item Z - $O_1$:
    \begin{equation}
        \left(L - \lambda\right)A + \lambda(-B_y \cos\theta  + B_x \sin\theta - O_{1y} \cos\theta + O_{1x} \sin\theta) = \bar{I_{o_1}} \cdot \alpha_1  
    \end{equation}
\end{enumerate}
\subsection*{Ecuaciones de movimiento eslabon O2-A}
La barra O2-A no presenta restriccion en la direccion x e y, ademas que su aceleracion angular es diferente de 0 y se le es impreso un par por parte del motor ($\tau$).
\begin{enumerate}
    \item X - $O_2$:
    \begin{equation}
        O_{2x} + A\sin\phi = m_r(\bar{a}_{o_2})_x
    \end{equation}

    \item Y - $O_2$:
    \begin{equation}
        O_{2y} - A\cos\phi - W_r= m_r(\bar{a}_{o_2})_y \quad \Rightarrow \quad O_{2y} - A\cos\phi = m_r((\bar{a}_{o_2})_y +g)
    \end{equation}

    \item Z - $O_2$:
    \begin{equation}
        \tau - \frac{r}{2}\cos({\phi - \theta}) + \frac{r}{2}(-O_{2y}\cos\phi + O_{2x}\sin\phi) = \bar{I_{o_2}} \cdot \alpha_2
    \end{equation}

\end{enumerate}

\subsection{Sistema reducido para resolución secuencial}
Dadas las 12 ecuaciones presenadas es posible reducir el sistema a 4 ecuaciones mediante remplazar $O_2x$ y $O_2y$ y resolver las ecuaciones que tienen solucion directa:
\begin{center}
\[
C = m_c \bigl(g - \bar{a}_c\bigr)
\]
\[
B_x = m_{BC}\,(\bar{a}_{BC})_x
\]
\[
B_y = m_{BC}\bigl(g + (\bar{a}_{BC})_y\bigr) + C
\]
\[
O_{2x} = m_r\,\bar{a}_{o_2x} - A\sin\theta
\]
\[
O_{2y} = m_r\bigl(\bar{a}_{o_2y} + g\bigr) + A\cos\theta
\]
\[
O_{1x} - A\sin\theta = m_{o1}\,\bar a_{O_{1x}} + B_x
\]
\[
O_{1y} + A\cos\theta = m_{o1}\bigl(\bar a_{O_{1y}} + g\bigr) + B_y
\]
\[
\lambda\sin\theta\,O_{1x} - \lambda\cos\theta\,O_{1y} + (L-\lambda)A
= \bar I_{o1}\,\alpha_1 - \lambda\bigl(B_x\sin\theta - B_y\cos\theta\bigr)
\]
\[
-2\cos(\phi-\theta)\,A + \frac{2}{r}\tau
= \frac{2}{r}\,\bar I_{o2}\,\alpha_2 + m_{o2}\bigl(g + \ddot{a}{O{2y}} - \ddot{a}{O{2x}}\bigr)
\]
\end{center}

\subsection{Potencia y eficiencia}
La potencia de entrada y salida se estiman como:
\begin{align*}
P_{in} &= \tau\,\omega_2, & P_{out} &= C\,\dot{y}_c, & \eta &= \frac{P_{out}}{P_{in}}\times 100\%.
\end{align*}

\section{Fase de Diseño Detallado}

\section{Análisis de los resultados teóricos}

Con el modelo matemático desarrollado en las secciones anteriores, se implementó un análisis numérico completo del mecanismo a lo largo de múltiples ciclos de operación. Los resultados presentados a continuación corresponden a las simulaciones realizadas con los parámetros geométricos y operativos especificados en la Sección 4, utilizando Python como plataforma de cálculo.

\subsection{Comportamiento cinemático del mecanismo}

\subsubsection{Variables geométricas fundamentales}

La distancia variable $L$ entre el pivote $O_1$ y el punto $A$ representa una magnitud clave en el análisis del mecanismo. Como se observa en la Figura \ref{fig:L_phi}, esta distancia varía de forma no lineal con el ángulo de entrada $\phi$, alcanzando valores entre aproximadamente 40 mm y 200 mm. Esta variación considerable confirma que el pasador deslizante recorre gran parte de la ranura en la barra $O_1B$ durante cada ciclo.

\begin{figure}[H]
\centering
\includegraphics[width=0.75\textwidth]{{salidas/L_vs_phi}.png}
\caption{Relación entre la distancia $L$ y el ángulo de entrada $\phi$ de la barra motriz.}
\label{fig:L_phi}
\end{figure}

La relación angular entre $\theta$ (ángulo de la barra oscilante) y $\phi$ (ángulo de la barra motriz) se muestra en la Figura \ref{fig:theta_t}. A pesar de que $\phi$ aumenta linealmente con el tiempo debido a la velocidad angular constante del motor ($\omega_2 = 2.0$ rad/s), el ángulo $\theta$ presenta una variación no lineal característica que genera la asimetría temporal del movimiento del martillo.

\begin{figure}[H]
\centering
\includegraphics[width=0.75\textwidth]{{salidas/theta_vs_t}.png}
\caption{Evolución temporal del ángulo $\theta$ de la barra oscilante $O_1B$.}
\label{fig:theta_t}
\end{figure}

El ángulo $\beta$ del eslabón $BC$ (Figura \ref{fig:beta_t}) controla la orientación del martillo respecto a la vertical. Este parámetro se mantiene dentro de límites seguros durante todo el ciclo, evitando configuraciones singulares donde $\cos\beta \to 0$, lo cual causaría indeterminaciones en las ecuaciones de velocidad angular $\omega_3$.

\begin{figure}[H]
\centering
\includegraphics[width=0.75\textwidth]{{salidas/beta_vs_t}.png}
\caption{Variación temporal del ángulo $\beta$ del eslabón $BC$.}
\label{fig:beta_t}
\end{figure}

\subsubsection{Cinemática del martillo}

El desplazamiento vertical del martillo $y_c$ (Figura \ref{fig:yc_t}) muestra el comportamiento cíclico del sistema. La carrera total del martillo (diferencia entre posiciones extremas) es de aproximadamente 254 mm, valor que resulta adecuado para aplicaciones de forja ligera o conformado mecánico.

\begin{figure}[H]
\centering
\includegraphics[width=0.75\textwidth]{{salidas/yc_vs_t}.png}
\caption{Posición vertical del martillo $y_c$ a lo largo del tiempo.}
\label{fig:yc_t}
\end{figure}

La velocidad vertical del martillo $\dot{y}_c$ (Figura \ref{fig:vyc_t}) evidencia claramente el comportamiento asimétrico del mecanismo de retorno rápido. Durante la fase de descenso (avance), el martillo se desplaza con velocidades relativamente bajas (del orden de 0.1--0.3 m/s), lo que permite un impacto controlado y una transferencia de energía más eficiente. Por el contrario, durante el ascenso (retorno), la velocidad aumenta significativamente (hasta aproximadamente 0.5 m/s), reduciendo así el tiempo muerto del ciclo y aumentando la productividad del sistema.

\begin{figure}[H]
\centering
\includegraphics[width=0.75\textwidth]{{salidas/dot_yc_vs_t}.png}
\caption{Velocidad vertical del martillo $\dot{y}_c$ mostrando la asimetría característica del ciclo.}
\label{fig:vyc_t}
\end{figure}

La aceleración vertical del martillo $\ddot{y}_c$ (Figura \ref{fig:ayc_t}) presenta picos considerables durante las transiciones del movimiento, especialmente en los puntos de inversión de la dirección. Estos valores de aceleración, que pueden alcanzar varios m/s², son críticos para el dimensionamiento estructural, ya que generan fuerzas inerciales significativas que deben ser soportadas por los eslabones y articulaciones del mecanismo.

\begin{figure}[H]
\centering
\includegraphics[width=0.75\textwidth]{{salidas/ac_vs_t}.png}
\caption{Aceleración vertical del martillo $\ddot{y}_c$ durante el ciclo completo.}
\label{fig:ayc_t}
\end{figure}

\subsubsection{Velocidades y aceleraciones angulares de los eslabones}

Las velocidades angulares de los eslabones $O_1B$ y $BC$ se comparan en la Figura \ref{fig:omega_comp}. Se observa que $\omega_3$ (velocidad angular del eslabón $BC$) alcanza valores considerablemente mayores que $\omega_1$ (velocidad angular de $O_1B$), lo cual es consecuencia directa de la geometría del mecanismo y la transmisión del movimiento. Los picos en $\omega_3$ ocurren cuando el ángulo $\beta$ se aproxima a configuraciones donde $\cos\beta$ es pequeño, amplificando el efecto de la rotación de la barra $O_1B$.

\begin{figure}[H]
\centering
\includegraphics[width=0.75\textwidth]{{salidas/comparacion_velocidades_angulares}.png}
\caption{Comparación de las velocidades angulares $\omega_1$ (eslabón $O_1B$) y $\omega_3$ (eslabón $BC$).}
\label{fig:omega_comp}
\end{figure}

Las aceleraciones angulares $\alpha_1$ y $\alpha_3$ (Figura \ref{fig:alpha_comp}) presentan variaciones importantes a lo largo del ciclo. Los valores máximos de estas aceleraciones generan momentos dinámicos considerables que deben tenerse en cuenta en el análisis de resistencia de los eslabones y en el diseño de las uniones articuladas. La amplitud de $\alpha_3$ es especialmente notable y está directamente relacionada con las fuerzas internas del eslabón $BC$.

\begin{figure}[H]
\centering
\includegraphics[width=0.75\textwidth]{{salidas/comparacion_aceleraciones_angulares}.png}
\caption{Comparación de las aceleraciones angulares $\alpha_1$ y $\alpha_3$.}
\label{fig:alpha_comp}
\end{figure}

\subsection{Análisis dinámico: fuerzas y reacciones}

\subsubsection{Reacciones en el pivote fijo $O_1$}

Las componentes horizontal y vertical de la reacción en el pivote fijo $O_1$ se presentan en las Figuras \ref{fig:O1x_t} y \ref{fig:O1y_t}. La componente horizontal $O_{1x}$ varía de forma oscilatoria y alcanza valores máximos durante las fases de mayor aceleración horizontal del sistema. La componente vertical $O_{1y}$ debe soportar no solo el peso de los eslabones y del martillo, sino también las cargas dinámicas generadas por las aceleraciones verticales. Estos valores son fundamentales para el diseño del soporte y la selección de los rodamientos del pivote.

\begin{figure}[H]
\centering
\includegraphics[width=0.75\textwidth]{{salidas/O1x_vs_t}.png}
\caption{Componente horizontal de la reacción en el pivote $O_1$.}
\label{fig:O1x_t}
\end{figure}

\begin{figure}[H]
\centering
\includegraphics[width=0.75\textwidth]{{salidas/O1y_vs_t}.png}
\caption{Componente vertical de la reacción en el pivote $O_1$.}
\label{fig:O1y_t}
\end{figure}

\subsubsection{Fuerza en el pasador deslizante $A$}

La magnitud de la fuerza en el pasador $A$ (Figura \ref{fig:A_t}) representa la carga transmitida entre la barra motriz $O_2A$ y la barra oscilante $O_1B$. Esta fuerza presenta variaciones cíclicas importantes y alcanza valores máximos que determinan los requisitos de resistencia del pasador y las dimensiones de la ranura. El diseño debe asegurar que el pasador no falle por fatiga ni provoque desgaste excesivo en la ranura durante la vida útil del mecanismo.

\begin{figure}[H]
\centering
\includegraphics[width=0.75\textwidth]{{salidas/A_vs_t}.png}
\caption{Magnitud de la fuerza en el pasador deslizante $A$.}
\label{fig:A_t}
\end{figure}

\subsubsection{Reacciones en el punto $B$}

Las reacciones en el punto $B$, donde se unen la barra $O_1B$ y el eslabón $BC$, se muestran en las Figuras \ref{fig:Bx_t} y \ref{fig:By_t}. La componente horizontal $B_x$ varía principalmente debido a las aceleraciones horizontales del eslabón $BC$, mientras que la componente vertical $B_y$ es particularmente significativa ya que debe transmitir el peso del martillo más las fuerzas dinámicas generadas durante el movimiento vertical. El perno o eje en $B$ debe dimensionarse para soportar la resultante de estas dos componentes.

\begin{figure}[H]
\centering
\includegraphics[width=0.75\textwidth]{{salidas/Bx_vs_t}.png}
\caption{Componente horizontal de la reacción en el punto $B$.}
\label{fig:Bx_t}
\end{figure}

\begin{figure}[H]
\centering
\includegraphics[width=0.75\textwidth]{{salidas/By_vs_t}.png}
\caption{Componente vertical de la reacción en el punto $B$.}
\label{fig:By_t}
\end{figure}

\subsubsection{Fuerza de contacto del martillo}

La fuerza de contacto $C$ ejercida por el martillo sobre la pieza de trabajo se presenta en la Figura \ref{fig:C_t}. Esta fuerza alcanza sus valores máximos durante la fase de impacto, cuando la velocidad del martillo se reduce bruscamente al entrar en contacto con la pieza. La magnitud y duración de esta fuerza determinan la efectividad del proceso de conformado o forjado. El valor máximo de $C$ es un parámetro crítico para la selección del material del martillo y para estimar la energía transferida en cada golpe.

\begin{figure}[H]
\centering
\includegraphics[width=0.75\textwidth]{{salidas/C_vs_t}.png}
\caption{Fuerza de contacto del martillo $C$ durante el ciclo de trabajo.}
\label{fig:C_t}
\end{figure}

\subsection{Requerimientos de par motor}

El par motor instantáneo $\tau$ necesario para mantener el movimiento uniforme del mecanismo se presenta en la Figura \ref{fig:tau_t}. Se observan variaciones significativas del par a lo largo del ciclo, con valores máximos que corresponden a las fases de mayor aceleración angular de los eslabones. El par promedio y los picos máximos son parámetros críticos para:

\begin{itemize}
    \item La selección del motor eléctrico y su potencia nominal
    \item El diseño del sistema de transmisión (reductor de velocidad)
    \item La evaluación de la necesidad de incorporar un volante de inercia para suavizar las fluctuaciones de par
\end{itemize}

La fluctuación considerable del par sugiere que la incorporación de un volante de inercia en el eje motor podría ser beneficiosa para reducir las demandas instantáneas sobre el motor y mejorar la eficiencia energética del sistema.

\begin{figure}[H]
\centering
\includegraphics[width=0.75\textwidth]{{salidas/tau_vs_t}.png}
\caption{Par motor $\tau$ requerido en función del tiempo para mantener $\omega_2$ constante.}
\label{fig:tau_t}
\end{figure}

\subsection{Síntesis de resultados}

El análisis numérico completo del mecanismo permite extraer las siguientes conclusiones principales:

\begin{enumerate}
    \item \textbf{Asimetría del ciclo:} El mecanismo cumple satisfactoriamente con el objetivo de producir un movimiento asimétrico. La relación entre los tiempos de avance y retorno es aproximadamente 1.5:1, lo que reduce significativamente el tiempo muerto del ciclo productivo.
    
    \item \textbf{Estabilidad cinemática:} Todas las variables cinemáticas ($\theta$, $\beta$, $\omega_1$, $\omega_3$, $\alpha_1$, $\alpha_3$) se mantienen dentro de rangos finitos y predecibles, evitando singularidades que podrían causar velocidades o aceleraciones infinitas.
    
    \item \textbf{Carrera útil:} El martillo alcanza una carrera de aproximadamente 254 mm, valor adecuado para operaciones de conformado ligero. Esta carrera puede ajustarse modificando los parámetros geométricos $R$, $K$ y $D$.
    
    \item \textbf{Fuerzas de reacción:} Los valores calculados de las fuerzas en los pivotes y articulaciones proporcionan información cuantitativa para el dimensionamiento de los componentes mecánicos (ejes, pernos, rodamientos, soldaduras).
    
    \item \textbf{Requerimientos de potencia:} Las fluctuaciones del par motor indican que podría ser ventajoso incorporar un volante de inercia para regularizar la demanda de potencia y reducir los picos de par.
    
    \item \textbf{Validación del modelo:} Los resultados obtenidos son coherentes con las ecuaciones teóricas desarrolladas y muestran un comportamiento físicamente realista del mecanismo.
\end{enumerate}

Estos resultados teóricos servirán como base de comparación para las pruebas experimentales que se realizarán con el prototipo físico, permitiendo validar el modelo matemático y ajustar los parámetros de diseño según sea necesario.

\section{Análisis sobre el funcionamiento del prototipo}

\section{Conclusiones y recomendaciones}

% ===== REFERENCIAS =====

\addcontentsline{toc}{section}{Referencias}
\begin{thebibliography}{99}


\bibitem{beer}
Ferdinand P. Beer, E. Russell Johnston Jr., William E. Clausen, "Dinámica para ingeniería", McGraw-Hill, última edición disponible.

\bibitem{hal-01715664}
Autor desconocido, “Vector loop approach for mechanism analysis,” HAL Open Science, 2018. Disponible en: https://hal.science/hal-01715664/document

\bibitem{stix2}
STIX Fonts Project, “STIX Two Math,” 2020. Disponible en: https://stixfonts.org

\end{thebibliography}

\end{document}
